
\section{Definizione dei Requisiti Utente}

Di seguito \`e illustrata la descrizione dettagliata dei requisiti utente.

\subsection{Requisiti Funzionali}
\label{sec:utente:funzionali}

I requisiti funzionali individuati per il sistema HBS sono i seguenti:

\begin{enumerate}
	\item \label{itm:utente:funzionali:iscrizione} Ogni utente correntista che ne faccia esplicita richiesta deve avere un account personale e deve potersi iscrivere  mediante apposite procedure (business case in figura \ref{fig:business_case_registration}).
	\item \label{itm:utente:funzionali:gestione-conto} Un utente registrato deve poter effettuare l'amministrazione corretta del proprio conto corrente (business case in figura \ref{fig:business_case_generic_operation}), per questa ragione è richiesto che:
	\begin{enumerate}
		\item \label{itm:utente:funzionali:gestione-conto:verifica-saldo} l'utente possa verificare in tempo reale:
			\begin{itemize}
				\item il suo saldo attuale;
				\item il suo saldo contabile;
				\item il suo saldo liquido;
			\end{itemize}
		\item \label{itm:utente:funzionali:gestione-conto:verifica-andamento} l'utente deve poter verificare in tempo reale l'andamento dei titoli azionari che possiede e, eventualmente, venderli;
		% \item \label{itm:utente:funzionali:gestione-conto:revisione} l'utente deve poter effettuare operazioni programmate di revisione dei conti o di pagamento;
		\item \label{itm:utente:funzionali:gestione-conto:operazioni} l'utente deve poter effettuare disposizioni generiche di pagamento;
		\item \label{itm:utente:funzionali:gestione-conto:operazioni-veloci} l'utente deve poter effettuare \emph{operazioni veloci}, ossia effettuare disposizioni inserendo pochi dati in una maschera preconfigurata, come:
			\begin{itemize}
				\item ricariche telefoniche;
				\item bonifici ordinari;
				\item bonifici SEPA;
				\item pagamento delle bollette.
			\end{itemize}
	\end{enumerate}
	\item \label{itm:utente:funzionali:storico} L'utente deve poter consultare lo storico delle transazioni eseguite, complete delle informazioni riguardanti la data, il destinatario, l'importo previsto e una causale descrittiva delle motivazioni;
	% \item \label{itm:utente:funzionali:resoconto} L'utente deve poter ottenere, in tempo reale, un resoconto delle transazioni che sono in corso da e verso il suo conto;
	\item \label{itm:utente:funzionali:dipendenti:accesso} I dipendenti di una banca, impiegati e manager, devono poter accedere al sistema mediante specifico e personale account;
	\item \label{itm:utente:funzionali:dipendenti:operazioni-veloci} I dipendenti di una banca devono poter creare i form per la gestione delle operazioni veloci;
	\item \label{itm:utente:funzionali:management:bidding:approvazione} I manager devono poter accettare o rifiutare le proposte di bidding per conti correnti, carte di credito e prestiti fatte dagli utenti che rientrano nell'insieme di proposte soggette a approvazione dei manager della banca.
	\item \label{itm:utente:funzionali:management:bidding:creazione} I manager devono poter gestire il bidding eventualmente offerto della banca, specificando in modo opportuno per conti correnti, carte di credito e prestiti i parametri che individuano i bid accettati automaticamente, soggetti a ulteriore approvazione, rifiutati automaticamente.
	I parametri di decisione comprendono i parametri di conti correnti, carte di credito e prestiti, e parametri riguardanti i clienti della banca.
	Alcuni possibili parametri sono:
		\begin{itemize}
			\item giacenza media mensile e annuale sui conti correnti dell'utente;
			\item affidibilit\`a dell'utente (in funzione, ad esempio, della frequenza con cui i suoi conti correnti sono andati in rosso);
			\item da quanto tempo il correntista \`e cliente presso la banca;
			\item eventuali altri parametri.
		\end{itemize}
	\item \label{itm:utente:funzionali:management:pacchetti-investimento:creazione} I manager devono poter creare \emph{pacchetti di investimento}.
		Un pacchetto di investimento contiene titoli azionari prelevati da un \emph{pool} di azioni acquistate dalla banca.
		Il pacchetto di investimento deve contenere le seguenti informazioni:
		\begin{itemize}
			\item andamento passato dei titoli contenuti nel pacchetto;
			\item valutazione di rischio prodotta dal manager che ha creato il pacchetto;
			\item costo del pacchetto;
			\item valore garantito di vendita del pacchetto;
			\item percentuale di guadagno della banca alla vendita del pacchetto.
		\end{itemize}
		I pacchetti di investimento vengono acquistati dai clienti della banca e formano un \emph{portafoglio di investimento}.
	\item \label{itm:utente:funzionali:management:pubblicita} I manager devono poter gestire in modo efficiente gli spazi pubblicitari offerti dalle pagine web del sistema, potendone arbitrariamente modificare i contenuti e indirizzando determinate pubblicità ad determinati insiemi di utenti aventi delle caratteristiche desiderate.
	\item \label{itm:utente:funzionali:notifiche} Il sistema HBS deve inviare ad ogni utente una notifica tramite SMS o e-mail (a scelta dell'utente) a seguito di ogni operazione tra le seguenti:
		\begin{itemize}
			\item ogni volta che si conclude una transazione da o verso il proprio conto corrente;
			\item ad ogni login e logout dall'account HBS;
			\item ad ogni \emph{operazione veloce} di pagamento.
		\end{itemize}
	Ogni notifica \`e abilitabile o disabilitabile singolarmente secondo le preferenze dell'utente.
	\item \label{itm:utente:funzionali:storico-dettagliato} Il sistema di Home Banking deve fornire ai suoi utenti (clienti della banca, dipendenti della banca, agenti di terze parti) la possibilit\`a di visualizzare uno storico dettagliato di ogni operazione (requisito funzionale) effettuata seguendo un opportuno sistema di privilegi.
\end{enumerate}

\subsection{Requisiti Non Funzionali}
\label{sec:utente:non-funzionali}

\begin{enumerate}
	\item \label{itm:utente:non-funzionali:logging} Ogni operazione effettuata da un utente (cliente della banca o dipendente della banca) sul sistema deve essere registrata in un log.
	Ogni log deve contenere quantomeno le seguenti informazioni:
	\begin{itemize}
		\item un'identificativo del tipo di operazione eseguita;
		\item il conto coinvolto nell'operazione;
		\item l'istante dell'operazione;
		\item informazioni riguardanti il terminale da cui \`e stata effettuata l'operazione.
	\end{itemize}

	\item \label{itm:utente:non-funzionali:uptime} Il sistema di Home Banking ha dei requisiti impliciti di disponibilit\`a, ovvero il sistema deve avere una percentuale minima garantita di \emph{uptime}, o tempo in cui il sistema \`e disponibile e utilizzabile dagli utenti.

	La percentuale di \emph{uptime} non comprende eventuale tempo di \emph{downtime} previsto periodicamente per motivi di manutenzione, ottimizzazione delle prestazioni, aggiornamento, etc.

	Le percentuali di \emph{uptime} garantito e di \emph{downtime} richiesto devono essere definite in funzione della quantit\`a di operazioni gestita dal sistema, ossia in funzione del numero di utenti attivi presso un particolare istituto bancario e della mole di transazioni effettuate quotidianamente, e in funzione delle caratteristiche delle macchine su cui il software viene installato presso un particolare istituto bancario.

	La percentuale di \emph{uptime} garantito e di \emph{downtime} richiesto deve quindi essere stabilita e personalizzata al momento della vendita del software ad un istituto bancario.

	\item \label{itm:utente:non-funzionali:usabilita} Il sistema di Home Banking ha dei requisiti impliciti di usabilit\`a, ovvero:
	\begin{enumerate}
		\item \label{itm:utente:non-funzionali:usabilita:clienti} deve essere utilizzabile senza particolare addestramento dai clienti della banca;
		\item \label{itm:utente:non-funzionali:usabilita:management} deve essere utilizzabile dopo addestramento minimo dai dipendenti della banca.
	\end{enumerate}
	Le operazioni disponibili per i clienti della banca (requisiti funzionali) devono essere facilmente utilizzabili e comprensibili.
	Ogni maschera per l'inserimento di informazioni deve essere corredata da opportuni testi brevi illustranti il tipo di informazioni richieste.
\end{enumerate}

\subsubsection{Requisiti di Sicurezza del Sistema}
\label{sec:utente:non-funzionali:sicurezza}

Il sistema di Home Banking presenta forti requisiti non funzionali di sicurezza.

\begin{enumerate}
	\item \label{itm:utente:non-funzionali:sicurezza:accesso} L'accesso al sistema di Home Banking da parte di un utente avviene a seguito di autenticazione dello stesso.
		Le credenziali di accesso devono essere trasmesse dal browser dell'utente al sistema di Home Banking utilizzando una connessione sicura.
	\item \label{itm:utente:non-funzionali:sicurezza:credenziali} Le credenziali d'accesso per un account sono:
		\begin{itemize}
			\item numero conto corrente;
			\item password fornita al correntista al momento della registrazione.
		\end{itemize}
	\item \label{itm:utente:non-funzionali:sicurezza:operazioni} Le operazioni effettuabili dall'utente (requisiti funzionali) sono partizionate in operazioni che richiedono un'ulteriore autenticazione tramite One Time Password e operazioni che \emph{non} richiedono ulteriore autenticazione.
	Ogni operazione \`e eseguibile solo dopo che l'utente ha effettuato l'accesso al sistema di Home Banking.

	Le seguenti operazioni non richiedono ulteriore autenticazione:
	\begin{itemize}
		\item visionare saldo contabile, disponibile e liquido;
		\item visionare uno storico delle transazioni effettuate;
		\item visionare informazioni riguardo le carte di credito collegate al conto (se presenti);
		\item effettuare ``operazioni veloci'' impostate attraverso un sistema di configurazione.
	\end{itemize}

	Ogni altra operazione richiede autenticazione tramite One Time Password.
	In particolare \`e richiesta autenticazione tramite One Time Password per:
	\begin{itemize}
		\item effettuare transazioni, come:
		\begin{itemize}
			\item bonifici ordinari, bonifici SEPA;
			\item ricariche carte prepatate e schede telefoniche;
			\item pagamento bollette, bollettini, tasse, etc;
			\item modificare le preferenze dell'utente, come frequenza delle notifiche, etc.
		\end{itemize}
		\item configurare operazioni veloci.
	\end{itemize}
\end{enumerate}

\subsection{Requisiti di Dominio}
\label{sec:utente:dominio}

La legislazione attuale richiede che gli organi di controllo finanziario come la Banca d'Italia e le forze dell'ordine possano accedere in lettura a tutte le informazioni salvate dal sistema di Online Banking.

\begin{enumerate}
	\item \label{itm:utente:dominio:vpn} Ogni sistema informatico nell'ambito bancario, e in particolare il software HBS in sviluppo, deve permettere l'accesso da remoto alla sua rete interna tramite il meccanismo di \emph{VPN}.
	HBS deve quindi fornire alle autorit\`a di controllo le credenziali e/o interfacce per accedere ai sistemi di \emph{data storage}.

	\item \label{itm:utente:dominio:forze-ordine} Le forze dell'ordine, dato un utente, devono poter visualizzare le seguenti informazioni:
        \begin{enumerate}
           	\item tutte le transazioni effettuati dall'utente;
            \item tutte le transazioni che hanno l'utente come destinatario;
	        \item dati anagrafici dell'utente;
    	    \item informazioni riguardo al terminale informatico dal quale l'\emph{account} dell'utente \`e stato acceduto in precedenza.
		\end{enumerate}
		In caso le informazioni siano ridondanti il sistema deve fornire alle ff. oo. un modo per confrontare le informazioni e fare il controllo di coerenza.
%TODO da qualche parte bisogna dire che un haxxor in genere non puo modificare tutti i log in maniera coerente, perche pensa a rubare i dindi invece di giocare ad uplink

	\item \label{itm:utente:dominio:controllo-finanziario} L'organo di controllo finanziario, dato un utente, deve poter visualizzare le seguenti informazioni:
        	\begin{enumerate}
	            \item tutte le transazioni effettuate dall'utente;
    	        \item tutte le transazioni che hanno l'utente come destinatario;
	            \item lo storico di Online Trading dell'utente.
    	    \end{enumerate}
	La normativa legale richiede che un sistema in grado di offrire funzionalit\`a di trading fornisca all'organo di controllo finanziario l'accesso allo storico delle operazioni, in particolare l'organo di controllo deve poter:
	\begin{enumerate}
    	\item visualizzare le informazioni dei pacchetti di Online Trading;
    	\item visualizzre lo storico del sistema di Online Trading.
    	%TODO non so se ci va visualizzare mutui e tassi
	\end{enumerate}

	\item \label{itm:utente:dominio:controllo-finanziario:bidding} Poich\'e il sistema di Bidding deve rispettare le normative legali, per esempio le leggi sull'usura bancaria, l'ente responsabile del controllo finanziario deve poter:
	\begin{enumerate}
    	\item accedere allo storico del sistema Bidding;
    	\item visualizzare i parametri entro i quali le proposte vengono accettate automaticamente;
    	\item visualizzare informazioni delle proposte accettate dal management;
    	\item visualizzare proposte negate dal management.
	\end{enumerate}

\end{enumerate}

%TODO conti sono observable, la gg.ff. puo definire trigger. .. . \ldots .. \ldots \ldots eas\ldots \ldots .. \ldots \ldots ..
