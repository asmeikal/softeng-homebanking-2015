\def\idDISPAG{REQ\_DISPAG\_F\_A\_1\,}
\def\shortidDISPAG{REQ\_F\_1\,}

\def\idISCRCORR{REQ\_ISCRCORR\_F\_A\_2\,}
\def\shortidISCRCORR{REQ\_F\_2\,}

\def\idCROPVEL{REQ\_CROPVEL\_F\_M\_3\,}
\def\shortidCROPVEL{REQ\_F\_3\,}

\def\idSECAUTH{REQ\_SECAUTH\_N\_A\_1\,}
\def\shortidSECAUTH{REQ\_N\_1\,}

\def\idDISOPVEL{REQ\_DISOPVEL\_F\_M\_4\,}
\def\shortidDISOPVEL{REQ\_F\_4\,}

\def\idAPPBID{REQ\_APPBID\_F\_M\_5\,}
\def\shortidAPPBID{REQ\_F\_5\,}

\def\idVERSAL{REQ\_VERSAL\_F\_A\_6\,}
\def\shortidVERSAL{REQ\_F\_6\,}

\def\idVERTIT{REQ\_VERTIT\_F\_M\_7\,}
\def\shortidVERTIT{REQ\_F\_7\,}

\def\idDIPACC{REQ\_DIPACC\_F\_A\_8\,}
\def\shortidDIPACC{REQ\_F\_8\,}

\def\idLOGOP{REQ\_LOGOP\_N\_A\_2\,}
\def\shortidLOGOP{REQ\_N\_2\,}

\def\idVERSTOR{REQ\_VERSTOR\_F\_A\_9\,}
\def\shortidVERSTOR{REQ\_F\_9\,}



\subsection{Specifica dei Requisiti Funzionali}

\subsubsection{\code{REQ\_F\_1} - Disposizioni di Pagamento}
\label{sec:sistema:funzionali:DISPAG}

\begin{ptable}{3}
\ptitlerow{Titolo}
\prow{
		Disposizioni di Pagamento
}
\pline
\pcells{ID & Tipologia & Priorit\`a}
\pcells{
	$\code{REQ\_DISPAG\_F\_A\_1}$ & Funzionale & Alta
}
\pline
\ptitlerow{Descrizione requisito}
\prow{
		Un utente registrato al sistema di Home Banking deve poter effettuare generiche disposizioni di pagamento. In particolare l'utente deve poter effettuare bonifici (p.~\pageref{glossario:bonifico}) e bonifici SEPA (p.~\pageref{glossario:bonifico-sepa}).
}
\pline
\ptitlerow{Origine requisito}
\prow{
		Requisiti utente (sez.\ \ref{sec:utente:funzionali}, req.\ \ref{itm:utente:funzionali:gestione-conto:operazioni}).
}
\pline
\ptitle{Input}
\pcell{1}{
		Numero di conto corrente, tipologia dell'operazione, somma di denaro da trasferire, conto corrente di destinazione nel formato opportuno.
}
\pline
\ptitle{Output}
\pcell{1}{
		Indicatore di successo dell'operazione.
}
\pline
\ptitlerow{Descrizione azione}
\prow{
		L'operazione specificata viene inoltrata da HBS al back-end della banca. HBS attende dal back-end informazioni riguardo il successo o meno dell'operazione.
}
%\pline
%\ptitlerow{Pre-condizioni}
%\prow{
%		L'utente ha accesso al conto corrente indicato. Il saldo contabile sul conto corrente indicato \`e maggiore o uguale alla somma di denaro da trasferire pi\`u eventuali spese di commissione.
%}
%\pline
%\ptitlerow{Post-condizioni}
%\prow{
%		Se il back-end della banca ha preso in carico l'operazione con successo, la gestione della stessa \`e stata passata al back-end. Altrimenti nessuna.
%}
%\pline
%\ptitlerow{Side-effects}
%\prow{
%		Il tentativo di operazione viene registrato nel log di HBS, insieme ad informazioni riguardo lo stato di presa in carico da parte del back-end della banca.
%}
\end{ptable}

\subsubsection{\code{REQ\_F\_2} - Iscrizione Nuovo Correntista}
\label{sec:sistema:funzionali:ISCRCORR}

\begin{ptable}{3}
\ptitlerow{Titolo}
\prow{
		Iscrizione Nuovo Correntista
}
\pline
\pcells{ID & Tipologia & Priorit\`a}
\pcells{
	$\code{REQ\_ISCRCORR\_F\_A\_2}$ & Funzionale & Alta
}
\pline
\ptitlerow{Descrizione requisito}
\prow{
		Il sistema deve permettere l'iscrizione a correntisti (p.~\pageref{glossario:correntista}) che non possiedano gi\`a un account HBS tramite un form compilabile e inviabile via web-browser.
}
\pline
\ptitlerow{Origine requisito}
\prow{
		Requisiti utente (sez.\ \ref{sec:utente:funzionali}, req.\ \ref{itm:utente:funzionali:iscrizione}).
}
\pline
\ptitle{Input}
\pcell{1}{
		Numero di conto corrente e codice fiscale del correntista.
}
\pline
\ptitle{Output}
\pcell{1}{
		Indicatore di avvenuta presa in carico della richiesta di iscrizione da parte del sistema di HBS.
}
\pline
\ptitlerow{Descrizione azione}
\prow{
		La richiesta di iscrizione viene presa in carico dal sistema e inserita nel database di HBS.
}
%\pline
%\ptitlerow{Pre-condizioni}
%\prow{
%		Il numero di conto corrente corrisponde a un conto corrente registrato presso l'istituto in questione. Il codice fiscale corrisponde alla persona fisica o giuridica cui il conto corrente \`e intestato. Non esiste un utente HBS associato al codice fiscale del correntista.
%}
%\pline
%\ptitlerow{Post-condizioni}
%\prow{
%		Viene creato un utente HBS associato al codice fiscale del correntista. Il nuovo utente non \`e abilitato all'accesso a HBS (v.\ requisiti \ref{itm:utente:funzionali:approvazione}). Il conto corrente associato al numero di conto corrente inserito viene abilitato all'uso di HBS.
%}
%\pline
%\ptitlerow{Side-effects}
%\prow{
%		La richiesta di iscrizione viene inserita nel database di HBS.
%}
\end{ptable}

\subsubsection{\code{REQ\_F\_3} - Creazione Operazioni Veloci}
\label{sec:sistema:funzionali:CROPVEL}

\begin{ptable}{3}
\ptitlerow{Titolo}
\prow{
		Creazione Operazioni Veloci
}
\pline
\pcells{ID & Tipologia & Priorit\`a}
\pcells{
	$\code{REQ\_CROPVEL\_F\_M\_3}$ & Funzionale & Media
}
\pline
\ptitlerow{Descrizione requisito}
\prow{
		I dipendenti della banca devono poter definire operazioni veloci utilizzabili dai clienti della stessa. Un'operazione veloce permette di effettuare operazioni ordinarie quali bonifici (p.~\pageref{glossario:bonifico}) o bonifici SEPA (p.~\pageref{glossario:bonifico-sepa}) inserendo un numero ridotto di argomenti rispetto a quelli normalmente necessari. Vedi anche requisito funzionale \ref{itm:utente:funzionali:gestione-conto:operazioni-veloci}.
}
\pline
\ptitlerow{Origine requisito}
\prow{
		Requisiti funzionali (sez.\ \ref{sec:utente:funzionali}, req.\ \ref{itm:utente:funzionali:dipendenti:operazioni-veloci}).
}
\pline
\ptitle{Input}
\pcell{1}{
		Nome dell'operazione veloce, parametri dell'operazione veloce, operazione tradizionale corrispondente, mapping dei parametri dell'operazione veloce nei parametri dell'operazione tradizionale.
}
\pline
\ptitle{Output}
\pcell{1}{
		Informazioni riguardo il successo dell'inserimento dell'operazione veloce nel database di HBS.
}
\pline
\ptitlerow{Descrizione azione}
\prow{
		Le informazioni fornite vengono inserite nel database di HBS contenente le informazioni sulle operazioni veloci configurate dalla banca.
}
%\pline
%\ptitlerow{Pre-condizioni}
%\prow{
%		Precondizioni.
%}
%\pline
%\ptitlerow{Post-condizioni}
%\prow{
%		Postcondizioni.
%}
%\pline
%\ptitlerow{Side-effects}
%\prow{
%		Side effects.
%}
\end{ptable}

\subsubsection{\code{REQ\_F\_4} - Disposizione Operazioni Veloci}
\label{sec:sistema:funzionali:DISOPVEL}

\begin{ptable}{3}
\ptitlerow{Titolo}
\prow{
		Disposizione Operazioni Veloci
}
\pline
\pcells{ID & Tipologia & Priorit\`a}
\pcells{
	$\code{REQ\_DISOPVEL\_F\_M\_4}$ & Funzionale & Media
}
\pline
\ptitlerow{Descrizione requisito}
\prow{
		Un utente registrato deve poter effettuare le operazioni veloci configurate dai dipendenti della banca.
}
\pline
\ptitlerow{Origine requisito}
\prow{
		Requisiti funzionali (sez.\ \ref{sec:utente:funzionali}, req.\ \ref{itm:utente:funzionali:gestione-conto:operazioni-veloci} e \ref{itm:utente:funzionali:dipendenti:operazioni-veloci}). Dipende da requisito sistema \shortidCROPVEL, sez.\ \ref{sec:sistema:funzionali:CROPVEL}.
}
\pline
\ptitle{Input}
\pcell{1}{
		Numero di conto corrente, tipologia di operazione veloce, argomenti dell'operazione veloce.
}
\pline
\ptitle{Output}
\pcell{1}{
		Indicatore di successo dell'operazione veloce.
}
\pline
\ptitlerow{Descrizione azione}
\prow{
		Il sistema traduce gli argomenti dell'operazione veloce negli argomenti necessari ad effettuare l'operazione tradizionale corrispondente, dopodich\'e effettua l'operazione tradizionale.
}
%\pline
%\ptitlerow{Pre-condizioni}
%\prow{
%		Precondizioni.
%}
%\pline
%\ptitlerow{Post-condizioni}
%\prow{
%		Postcondizioni.
%}
%\pline
%\ptitlerow{Side-effects}
%\prow{
%		Side effects.
%}
\end{ptable}

\subsubsection{\code{REQ\_F\_5} - Creazione Bidding}
\label{sec:sistema:funzionali:CREABID}

\begin{ptable}{3}
\ptitlerow{Titolo}
\prow{
		Creazione Bidding
}
\pline
\pcells{ID & Tipologia & Priorit\`a}
\pcells{
	$\code{REQ\_CREABID\_F\_A\_5}$ & Funzionale & Alta
}
\pline
\ptitlerow{Descrizione requisito}
\prow{
		Un manager della banca deve poter creare una regola di bidding, impostando i parametri di scelta in maniera opportuna per accettare o rifiutare automaticamente un bid, e per inviarlo al manager per approvazione altrimenti.
}
\pline
\ptitlerow{Origine requisito}
\prow{
		Requisiti utente (sez.~\ref{sec:utente:funzionali}, req.~\ref{itm:utente:funzionali:management:bidding:creazione}).
}
\pline
\ptitle{Input}
\pcell{1}{
		Una regola di bidding.
}
\pline
\ptitle{Output}
\pcell{1}{
		Descrizione del successo dell'azione.
}
\pline
\ptitlerow{Descrizione azione}
\prow{
		La regola di bidding proposta viene aggiunta al sistema di HBS.
}
%\pline
%\ptitlerow{Pre-condizioni}
%\prow{
%		Precondizioni.
%}
%\pline
%\ptitlerow{Post-condizioni}
%\prow{
%		Postcondizioni.
%}
%\pline
%\ptitlerow{Side-effects}
%\prow{
%		Side effects.
%}
\end{ptable}

\subsubsection{\code{REQ\_F\_6} - Approvazione Bidding}
\label{sec:sistema:funzionali:APPBID}

\begin{ptable}{3}
\ptitlerow{Titolo}
\prow{
		Approvazione Bidding
}
\pline
\pcells{ID & Tipologia & Priorit\`a}
\pcells{
	$\code{REQ\_APPBID\_F\_A\_6}$ & Funzionale & Alta
}
\pline
\ptitlerow{Descrizione requisito}
\prow{
		I manager della banca devono poter approvare o rifiutare i bid che non soddisfino n\'e i requisiti di rifiuto automatico n\'e i requisiti di approvazione automatica.
}
\pline
\ptitlerow{Origine requisito}
\prow{
		Requisiti funzionali (sez.~\ref{sec:utente:funzionali}, req.~\ref{itm:utente:funzionali:management:bidding:approvazione}).
}
\pline
\ptitle{Input}
\pcell{1}{
		Un bid.
}
\pline
\ptitle{Output}
\pcell{1}{
		Un valore indicante l'approvazione o il rifiuto del bid.
}
\pline
\ptitlerow{Descrizione azione}
\prow{
		Il sistema presenta al manager il bid selezionato, assieme a informazioni riguardo la storia finanziaria presso l'istituto dell'utente autore del bid. Il manager approva o rifiuta il bid.
}
%\pline
%\ptitlerow{Pre-condizioni}
%\prow{
%		Precondizioni.
%}
%\pline
%\ptitlerow{Post-condizioni}
%\prow{
%		Postcondizioni.
%}
%\pline
%\ptitlerow{Side-effects}
%\prow{
%		Side effects.
%}
\end{ptable}

\subsubsection{\code{REQ\_F\_7} - Verifica Saldo}
\label{sec:sistema:funzionali:VERSAL}

\begin{ptable}{3}
\ptitlerow{Titolo}
\prow{
		Verifica Saldo
}
\pline
\pcells{ID & Tipologia & Priorit\`a}
\pcells{
	$\code{REQ\_VERSAL\_F\_A\_7}$ & Funzionale & Alta
}
\pline
\ptitlerow{Descrizione requisito}
\prow{
		Un utente registrato al sistema di Home Banking e che abbia effettuato l'accesso deve poter visualizzare in tempo reale il saldo contabile (p.~\pageref{glossario:saldo-contabile}), il saldo attuale (p.~\pageref{glossario:saldo-attuale}) e il saldo liquido (p.~\pageref{glossario:saldo-liquido}) dei suoi conti correnti.
}
\pline
\ptitlerow{Origine requisito}
\prow{
		Requisiti utente (sez.\ \ref{sec:utente:funzionali}, req.\ \ref{itm:utente:funzionali:gestione-conto:verifica-saldo}).
}
\pline
\ptitle{Input}
\pcell{1}{
		Identificativo dell'utente e numero di conto corrente.
}
\pline
\ptitle{Output}
\pcell{1}{
		Saldo contabile, saldo attuale e saldo liquido del conto corrente.
}
\pline
\ptitlerow{Descrizione azione}
\prow{
		Calcola il saldo contabile, il saldo attuale e il saldo liquido del conto corrente indicato.
}
%\pline
%\ptitlerow{Pre-condizioni}
%\prow{
%		L'utente ha accesso al conto corrente indicato.
%}
%\pline
%\ptitlerow{Post-condizioni}
%\prow{
%		Nessuna.
%}
%\pline
%\ptitlerow{Side-effects}
%\prow{
%		Nessuno.
%}
\end{ptable}

\subsubsection{\code{REQ\_F\_8} - Verifica Andamento Titoli Azionari}
\label{sec:sistema:funzionali:VERTIT}

\begin{ptable}{3}
\ptitlerow{Titolo}
\prow{
		Verifica Andamento Titoli Azionari
}
\pline
\pcells{ID & Tipologia & Priorit\`a}
\pcells{
	$\code{REQ\_VERTIT\_F\_M\_8}$ & Funzionale & Media
}
\pline
\ptitlerow{Descrizione requisito}
\prow{
		Un utente registrato al sistema di Home Banking e che abbia effettuato l'accesso deve poter visualizzare in tempo reale l'andamento dei propri titoli azionari.
}
\pline
\ptitlerow{Origine requisito}
\prow{
		Requisiti utente (sez.\ \ref{sec:utente:funzionali}, req.\ \ref{itm:utente:funzionali:gestione-conto:verifica-andamento}).
}
\pline
\ptitle{Input}
\pcell{1}{
		Numero del conto corrente.
}
\pline
\ptitle{Output}
\pcell{1}{
		Andamento storico delle azioni.
}
\pline
\ptitlerow{Descrizione azione}
\prow{
		Il sistema di HBS raccoglie gli identificativi dei titoli azionari di cui il correntista dispone e recupera l'andamento delle azioni dal back-end della banca.
}
%\pline
%\ptitlerow{Pre-condizioni}
%\prow{
%		L'utente ha accesso al conto corrente indicato.
%}
%\pline
%\ptitlerow{Post-condizioni}
%\prow{
%		Nessuna.
%}
%\pline
%\ptitlerow{Side-effects}
%\prow{
%		Nessuno.
%}
\end{ptable}

\subsubsection{\code{REQ\_F\_9} - Bidding Utente}
\label{sec:sistema:funzionali:USRBID}

\begin{ptable}{3}
\ptitlerow{Titolo}
\prow{
		Bidding Utente
}
\pline
\pcells{ID & Tipologia & Priorit\`a}
\pcells{
	$\code{REQ\_USRBID\_F\_A\_9}$ & Funzionale & Alta
}
\pline
\ptitlerow{Descrizione requisito}
\prow{
		Un utente di HBS deve poter effettuare bidding per conti correnti, carte di credito e prestiti, in caso siano state configurate delle regole di bidding da parte dei manager della banca.
}
\pline
\ptitlerow{Origine requisito}
\prow{
		Requisiti utente (sez.~\ref{sec:utente:funzionali}, req.~\ref{itm:utente:funzionali:bidding:utente}).
}
\pline
\ptitle{Input}
\pcell{1}{
		Una proposta di bidding per un conto corrente, una carta di credito o un prestito.
}
\pline
\ptitle{Output}
\pcell{1}{
		Descrizione del successo del bid, ossia indicatore di approvazione, presa in carico dal management, o rifiuto della proposta.
}
\pline
\ptitlerow{Descrizione azione}
\prow{
		Per ogni regola di bidding relativa all'oggetto in questione (conto corrente, carta di credito o prestito), il.sistema controlla che questa non rifiuti il bid ricevuto dall'utente. Se almeno una regola rifiuta il bid, questo viene rifiutato. Se tutte le regole approvano il bid, questo viene approvato. Se nessuna regola rifiuta il bid e non tutte le regole approvano il bid, questo viene memorizzato nel sistema di Home Banking come ``bid in attesa di approvazione da parte del management''.
}
%\pline
%\ptitlerow{Pre-condizioni}
%\prow{
%		Precondizioni.
%}
%\pline
%\ptitlerow{Post-condizioni}
%\prow{
%		Postcondizioni.
%}
%\pline
%\ptitlerow{Side-effects}
%\prow{
%		Side effects.
%}
\end{ptable}

\subsubsection{\code{REQ\_F\_10} - Accesso Account Dipendenti}
\label{sec:sistema:funzionali:DIPACC}

\begin{ptable}{3}
\ptitlerow{Titolo}
\prow{
		Accesso Account Dipendenti
}
\pline
\pcells{ID & Tipologia & Priorit\`a}
\pcells{
	$\code{REQ\_DIPACC\_F\_A\_10}$ & Funzionale & Alta
}
\pline
\ptitlerow{Descrizione requisito}
\prow{
		I dipendenti della banca (manager e impiegati) possono accedere alle pagine di amministrazione del sistema di Home Banking.
}
\pline
\ptitlerow{Origine requisito}
\prow{
		Requisiti funzionali (sez.~\ref{sec:utente:funzionali}, req.~\ref{itm:utente:funzionali:dipendenti:accesso}).
}
\pline
\ptitle{Input}
\pcell{1}{
		Input del requisito.
}
\pline
\ptitle{Output}
\pcell{1}{
		Output del requisito.
}
\pline
\ptitlerow{Descrizione azione}
\prow{
		Descrizione dell'algoritmo.
}
%\pline
%\ptitlerow{Pre-condizioni}
%\prow{
%		Precondizioni.
%}
%\pline
%\ptitlerow{Post-condizioni}
%\prow{
%		Postcondizioni.
%}
%\pline
%\ptitlerow{Side-effects}
%\prow{
%		Side effects.
%}
\end{ptable}

\subsubsection{\code{REQ\_F\_11} - Verifica Storico}
\label{sec:sistema:funzionali:VERSTOR}

\begin{ptable}{3}
\ptitlerow{Titolo}
\prow{
		Verifica Storico
}
\pline
\pcells{ID & Tipologia & Priorit\`a}
\pcells{
	$\code{REQ\_VERSTOR\_F\_A\_11}$ & Funzionale & Alta
}
\pline
\ptitlerow{Descrizione requisito}
\prow{
		Un utente deve poter verificare lo storico delle transazioni eseguite dal suo conto in un determinato periodo, complete di informazioni su: \begin{itemize} \item data dell'operazione; \item destinatario dell'operazione; \item causale dell'operazione; \item importo dell'operazione. \end{itemize}
}
\pline
\ptitlerow{Origine requisito}
\prow{
		Requisiti utente (sez.~\ref{sec:utente:funzionali}, req.~\ref{itm:utente:funzionali:storico}).
}
\pline
\ptitle{Input}
\pcell{1}{
		Numero del conto corrente, periodo di riferimento.
}
\pline
\ptitle{Output}
\pcell{1}{
		Elenco delle operazioni effettuate nel periodo di riferimento fornito.
}
\pline
\ptitlerow{Descrizione azione}
\prow{
		Il sistema recupera dal back-end della banca le informazioni richieste dall'utente.
}
%\pline
%\ptitlerow{Pre-condizioni}
%\prow{
%		Precondizioni.
%}
%\pline
%\ptitlerow{Post-condizioni}
%\prow{
%		Postcondizioni.
%}
%\pline
%\ptitlerow{Side-effects}
%\prow{
%		Side effects.
%}
\end{ptable}

\clearpage

\subsection{Specifica dei Requisiti Non Funzionali}

\subsubsection{\code{REQ\_N\_1} - Sicurezza Autenticazione Utenti}
\label{sec:sistema:non-funzionali:SECAUTH}

\begin{ptable}{3}
\ptitlerow{Titolo}
\prow{
		Sicurezza Autenticazione Utenti
}
\pline
\pcells{ID & Tipologia & Priorit\`a}
\pcells{
	$\code{REQ\_SECAUTH\_N\_A\_1}$ & Non Funzionale & Alta
}
\pline
\ptitlerow{Descrizione requisito}
\prow{
		L'autenticazione degli utenti del sistema di Home Banking (clienti della banca e dipendenti della banca) deve avvenire a seguito dell'invio delle credenziali di login attraverso una connessione sicura.
}
\pline
\ptitlerow{Origine requisito}
\prow{
		Requisiti non funzionali (sez. \ref{sec:utente:non-funzionali:sicurezza}, req.\ \ref{itm:utente:non-funzionali:sicurezza:accesso}).
}
%\pline
%\ptitlerow{Input}
%\prow{
%		Input del requisito.
%}
%\pline
%\ptitlerow{Output}
%\prow{
%		Output del requisito.
%}
%\pline
%\ptitlerow{Descrizione azione}
%\prow{
%		Descrizione dell'algoritmo.
%}
%\pline
%\ptitlerow{Pre-condizioni}
%\prow{
%		Precondizioni.
%}
%\pline
%\ptitlerow{Post-condizioni}
%\prow{
%		Postcondizioni.
%}
%\pline
%\ptitlerow{Side-effects}
%\prow{
%		Side effects.
%}
\end{ptable}

\subsubsection{\code{REQ\_N\_2} - Logging operazioni}
\label{sec:sistema:non-funzionali:LOGOP}

\begin{ptable}{3}
\ptitlerow{Titolo}
\prow{
		Logging operazioni
}
\pline
\pcells{ID & Tipologia & Priorit\`a}
\pcells{
	$\code{REQ\_LOGOP\_N\_A\_2}$ & Non Funzionale & Alta
}
\pline
\ptitlerow{Descrizione requisito}
\prow{
		Il sistema deve effettuare il logging di ogni operazione effettuata da un utente (cliente della banca o dipendente dell'istituto bancario).
}
\pline
\ptitlerow{Origine requisito}
\prow{
		Requisiti non funzionali (sez.\ \ref{sec:utente:non-funzionali}, req.\ \ref{itm:utente:non-funzionali:logging}).
}
%\pline
%\ptitlerow{Input}
%\prow{
%		Informazioni su un'operazione.
%}
%\pline
%\ptitlerow{Output}
%\prow{
%		Nessuno.
%}
%\pline
%\ptitlerow{Descrizione azione}
%\prow{
%		Il sistema registra in un apposito database le seguenti informazioni: \begin{itemize} \item identificativo univoco dell'operazione; \item tipologia dell'operazione; \item numero di conto corrente coinvolto nell'operazione; \item eventuale input dell'operazione; \item istante dell'operazione; \item identificativo dell'utente che ha effettuato l'operazione; \item informazioni riguardanti il dispositivo da cui l'operazione \`e stata effettuata, come: \begin{itemize} \item indirizzo IP del dispositivo; \item browser del dispositivo; \item sistema operativo del dispositivo. \end{itemize} \end{itemize}
%}
%\pline
%\ptitlerow{Pre-condizioni}
%\prow{
%		Nessuna.
%}
%\pline
%\ptitlerow{Post-condizioni}
%\prow{
%		Nessuna.
%}
%\pline
%\ptitlerow{Side-effects}
%\prow{
%		Le informazioni indicate sono state inserite nel database di logging di HBS.
%}
\end{ptable}

\clearpage

\subsection{Specifica dei Requisiti di Dominio}