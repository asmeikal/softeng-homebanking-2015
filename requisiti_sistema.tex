
\section{Specifica dei requisiti di sistema}

Di seguito \`e esposta la specifica dei requisiti di sistema.

Ciascun requisito \`e specificato su una scheda seguendo l'esempio illustrato da \cite[p. 97]{sommerville}.
In particolare ciascuna scheda conterr\`a i seguenti campi:
\begin{itemize}
	\item ID del requisito
	\item Tipo del requisito
	\item Titolo del requisito
	\item Descrizione del requisito
	\item Origine del requisito
	\item Priorit\`a del requisito
\end{itemize}
I requisiti funzionali conterranno inoltre i seguenti campi:
\begin{itemize}
	\item Input del requisito funzionale
	\item Origine dell'input
	\item Output del requisito funzionale
	\item Destinazione dell'output
	\item Descrizione dell'azione
	\item Pre-condizioni
	\item Post-condizioni
	\item Side-effects
\end{itemize}

Identifichiamo i requisiti del sistema con la codifica ottenuta concatenando i seguenti elementi, usando un \emph{underscore} come separatore:
\begin{itemize}
	\item il prefisso comune \code{REQ}
	\item una sequenza di 4-6 lettere (identificativo mnemonico)
	\item un identificativo di priorit\`a (in ordine decrescente \code{H}, \code{M}, \code{L})
	\item un identificativo della tipologia (\code{F} per requisiti funzionali, \code{N} per requisiti non funzionali, \code{D} per requisiti di dominio)
	\item un intero strettamente positivo (unico per ogni requisito).
\end{itemize}
In contesti in cui \`e necessario essere concisi \`e possibile identificare i requisiti utilizzando unicamente il prefisso \code{REQ} e l'intero assegnato al requisito.

\subsection{Specifica dei requisiti funzionali}

\subsubsection{\code{REQ\_1} - Iscrizione nuovo correntista}

\begin{ptable}{3}
	\ptitlerow{\bf Iscrizione nuovo correntista}
	\pcells{ID & Tipologia & Priorit\`a}
	\pcells{$\code{REQ\_ISCR\_H\_F\_1}$ & Funzionale & Alta}
	\pline
	\ptitlerow{Descrizione requisito}
	\prow{
		Il sistema deve permettere l'iscrizione a correntisti (p.~\pageref{correntista}) che non possiedano già un account HBS tramite un form compilabile e inviabile via web-browser.
	}
	\pline
	\ptitlerow{Origine requisito}
	\prow{
		Requisiti utente \ref{itm:iscrizione_utente}.
	}
	\pline
	\ptitlerow{Input}
	\prow{
		Numero di conto corrente e codice fiscale del correntista.
	}
	\pline
	\ptitlerow{Output}
	\prow{
		Indicatore di avvenuta presa in carico della richiesta di iscrizione da parte del sistema di HBS.
	}
	\pline
	\ptitlerow{Descrizione azione}
	\prow{
		La richiesta di iscrizione viene presa in carico dal sistema e inserita nel database di HBS.
	}
	\pline
	\ptitlerow{Pre-condizioni}
	\prow{
		Il numero di conto corrente corrisponde a un conto corrente registrato presso l'istituto in questione.
		Il codice fiscale corrisponde alla persona fisica o giuridica cui il conto corrente è intestato.
		Non esiste un utente HBS associato al codice fiscale del correntista.
	}
	\pline
	\ptitlerow{Post-condizioni}
	\prow{
		Viene creato un utente HBS associato al codice fiscale del correntista.
		Il nuovo utente non è abilitato all'accesso a HBS (v. requisito approvazione).
		Il conto corrente associato al numero di conto corrente inserito viene abilitato all'uso di HBS.
	}
	\pline
	\ptitlerow{Side-effects}
	\prow{
		La richiesta di iscrizione viene inserita nel database di HBS.
	}
\end{ptable}

%TODO registrazione conto corrente a utente già iscritto HBS
