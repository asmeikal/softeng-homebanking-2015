\subsection{Diagramma di Gantt}

Rappresentiamo la temporizzazione delle attivit\`a attraverso le fasi del progetto e teniamo traccia del loro stato di esecuzione con una serie di diagrammi di Gantt.
Il tempo di esecuzione richiesto da ciascuna attivit\`a non rispecchia perfettamente i tempi previsti: alcune attivit\`a, come ad esempio la creazione degli use case o del glossario, sono state protratte nel tempo per procedere di pari passo con attivit\`a da cui dipendono.

%\begin{figure*}[tbh]
%	\centering
%     \begin{ganttchart}[
%			title/.style={fill={color1!10}},
%			title label font=\color{color2},
%			title left shift=.1,
%			title right shift=-.1,
%			title top shift=.05,
%			hgrid,
%			bar label font = \scriptsize,
%			group label font = \scriptsize,
%			milestone label font = \scriptsize,
%			x unit=1cm,
%%			milestone height = 0.2,
%			today=2016-04,
%			today label = {},
%			today rule/.style = {draw={color1!60}},
%			progress label text = {\pgfmathprintnumber[precision=0, verbatim]{#1}\%},
%			inline,
%			milestone inline label node/.append style={left=1mm},
%			time slot format={isodate-yearmonth},
%			compress calendar,
%		]{2016-02}{2016-08}
%           \gantttitlecalendar{year, month} \\
%%           \ganttbar{Sviluppo proposta}{2015-11-01}{2015-12-01} \\
%           \ganttbar{Inception}{2016-02}{2016-04} \\
%           \ganttmilestone{Conclusione inception}{2016-04} \\
%           \ganttbar{Elaboration}{2016-05}{2016-06} \\
%           \ganttmilestone{Conclusione elaboration}{2016-06} \\
%           \ganttbar{Construction}{2016-07}{2016-08} \\
%           \ganttmilestone{Conclusione construction}{2016-08}
%      \end{ganttchart}
%\end{figure*}

\begin{figure*}
\centering
\begin{ganttchart}[
			title/.style={fill={color1!10}},
			title label font=\color{color2},
			title left shift=.1,
			title right shift=-.1,
			title top shift=.05,
			hgrid,
			vgrid,
			bar label font = \scriptsize,
			group label font = \scriptsize,
			milestone label font = \scriptsize,
			x unit=2.2mm,
			milestone height = 0.2,
			today=2016-04-04,
			today label = {},
			today rule/.style = {draw={color1!60}},
			progress label text = {\pgfmathprintnumber[precision=0, verbatim]{#1}\%},
			inline,
			milestone inline label node/.append style={left=1mm},
			time slot format={isodate},
		]{2016-02-22}{2016-04-30}
	\gantttitlecalendar{year, month} \\
	\ganttgroup[progress=today]{Analisi del contesto}{2016-02-22}{2016-04-10} \\
	\ganttbar[progress=100]{Raccolta informazioni contesto}{2016-02-22}{2016-03-22} \\
	\ganttbar[progress=100]{Studio business cases}{2016-03-01}{2016-03-22} \\
	\ganttbar[progress=80]{Studio di fattibilit\`a}{2016-03-2}{2016-04-06} \\
	\ganttbar[progress=60]{Stesura documento analisi contesto}{2016-03-14}{2016-04-10} \\
	\ganttgroup[progress=today]{Raccolta requisiti}{2016-03-09}{2016-04-22} \\
	\ganttbar[progress=80]{Definizione requisiti}{2016-03-09}{2016-04-18} \\
	\ganttbar[progress=20]{Specifica requisiti}{2016-03-28}{2016-04-18} \\
	\ganttbar[progress=60]{Creazione glossario}{2016-03-16}{2016-04-18} \\
	\ganttbar[progress=10]{Definizione use-case principali}{2016-03-20}{2016-04-18} \\
	\ganttbar[progress=50]{Stesura documento requisiti}{2016-03-16}{2016-04-22} \\
	\ganttgroup[progress=today]{Pianificazione progetto}{2016-03-25}{2016-04-26} \\
	\ganttbar[progress=40]{Definizione milestone}{2016-03-25}{2016-04-10} \\
	\ganttbar[progress=70]{Analisi dei rischi}{2016-03-25}{2016-04-10} \\
	\ganttbar[progress=50]{Analisi dei costi}{2016-03-29}{2016-04-16} \\
	\ganttbar[progress=60]{Stesura piano di progetto}{2016-03-25}{2016-04-25} \\
	\ganttmilestone[progress=today]{Prima revisione capo progetto}{2016-04-05} \\
	\ganttmilestone[progress=today]{Termine inception}{2016-04-30}
\end{ganttchart}
\caption{Diagramma di Gantt della fase di Inception.}
\end{figure*}

