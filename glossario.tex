\section{Glossario}
%(versione 1 : sono omesse definizioni banali e/o implicite)

\paragraph{Audit di sicurezza} 
	operazioni di controllo del sistema effettuate periodicamente per controllare che non ci siano sate violazioni della politica di sicurezza adottata
\paragraph{Banca D'Italia}
	 Banca centrale della Repubblica Italiana, avente funzione sia di vigilanza su banche, istituti di credito, intermediari finanziari sia di principale controllore in materia di antiriciclaggio: insieme alla CONSOB, deve avere libero accesso, entro i termini previsti dalla legge, ai dati contenuti nei database di qualsiasi istituto finanziario che operi sul territorio italiano. \cite{banca_italia}
\paragraph{Bonifico}     
	operazione bancaria mediante la quale si mette a disposizione di una persona o le si accredita una somma di denaro per ordine e conto di altri
\paragraph{Bonifico Sepa}
	bonifico (v.bonifico) accreditabile ad un destinatario ubicato oltre i confini nazionali nell'area SEPA.
\paragraph{Conto Corrente (cc)}
	indica un deposito di danaro effettuato dal possesore del bene, detto \emph{correntista}, in una banca o istituto di credito
\paragraph{Codice IBAN}
	L'International Bank Account Number è uno standard internazionale utilizzato per identificare un'utenza bancaria e/o operazione associata ad un conto corrente 
\paragraph{Codice BIC/SWIFT}
	standard che definisce i \emph{bank identifier codes} (codici d'identificazione bancaria) approvato dall'International Organization for Standardization (ISO). Questi codici vengono utilizzati per i trasferimenti di denaro tra banche, specialmente nelle transazioni internazionali, per le quali è spesso ancora necessario nonostante l'entrata in vigore dell'IBAN. \cite{bic_wiki}
\paragraph{Commissione Nazionale per le societ\`a e la borsa (CONSOB)}
	è un'autorità amministrativa indipendente, dotata di personalità giuridica e piena autonomia la cui attività è rivolta alla tutela degli investitori, all'efficienza, alla trasparenza e allo sviluppo del mercato mobiliare italiano. \cite{consob_wiki}
\paragraph{Fondo comune di investimento}
	è un istituto d'intermediazione finanziaria mediante il quale è possibile partecipare, investita una determinata quota di danaro, alla gestione e alla spartizione di dividendi prodotti da un determinato \emph{bene mobiliare} nel tempo. La \emph{banca depositaria} ne custodisce materialmente i titoli e ne tiene in cassa le disponibilità liquide. Le banche hanno inoltre un ruolo di controllo sulla legittimità delle attività del fondo sulla base di quanto prescritto dalle norme della Banca d'Italia e dal regolamento del fondo stesso
\paragraph{Istituto di credito}
	organismo che svolge simultaneamente l’attività di raccolta di risorse finanziarie e di concessione del credito per proprio contoa terzi 
\paragraph{Open Bank Project (OBP)}
	è una API open source che permette a banche ed istituti di credito di creare un interfaccia utente di ampia portata e fruibilità. Essendo un sistema molto versatile e largamente riadattabile, si presta molto a definire un vero e proprio \emph{standard di interfaccia}, ossia a definire un canone per la creazione di interfacce rivolte e all'utenza bancaria generica e agli organismi deputati al controllo bancario. \cite{obp}
\paragraph{Portafoglio valori/titoli azionari}
	è l'insieme dei diversi titoli finanziari e/o fondi d'investimento che l'utente bancario generico può possedere.Ogni titolo e/o fondo acquisito viene inserito nel portafoglio.
\paragraph{Refactoring}
	processo di riutilizzazzione di codice già scritto in precedenza, senza doverne generare di nuovo
\paragraph{SEPA}
	La SEPA (Single Euro Payments Area) è l’area unica in cui i cittadini, le imprese e gli enti, possono eseguire e ricevere pagamenti in Euro, all’interno dei confini nazionali e tra i paesi diversi che compongono l’area SEPA con condizioni di base, diritti ed obblighi uniformi tra i paesi stessi. 
\paragraph{Time-based One Time Password (TOTP)}
	è un algoritmo che calcola una \emph{One-Time password} combinando mediante una funzione hash una chiave segreta condivisa ed il tempo corrente. \cite{totprfc}
\paragraph{Trading online}
	pratica uguale a quella del \emph{trading} bancario classico mediante aiuto di personalità con competenze specifiche(v.broker), effettuata però in rete, disponendo cioè di opportuni strumenti software per il monitoraggio di mercati azionari nazionali e internazionali e  per il controllo completo e \emph{real time} del proprio portafoglio azionario.
