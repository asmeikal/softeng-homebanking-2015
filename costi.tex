\section{Analisi dei costi}

La stima dei costi di realizzazione del progetto viene effettuata con la tecnica degli \emph{Use Case Points}.

Gli Use Case Points del progetto sono dati dalla somma di due addendi:
\begin{description}
	\item[Unadjusted Use Case Weight]
		Misura il numero e la complessit\`a degli Use Case del sistema.
	\item[Unadjusted Actor Weight]
		Misura il numero e la complessit\`a degli attori del sistema.
\end{description}
Il valore ottenuto dalla somma di \code{UUCW} e \code{UAW} viene moltiplicato per due fattori:
\begin{description}
	\item[Technical Complexity Factor]
		Misura la complessit\`a tecnica del sistema.
	\item[Environmental Complexity Factor]
		Misura la complessit\`a dell'ambiente in cui il sistema viene sviluppato.
\end{description}

\subsection{Unadjusted Use Case Weight}

Il \code{UUCW} assegna un punteggio a ciascun use case del sistema secondo la complessit\`a dello use case.

Il \code{UUCW} sar\`a determinato a individuazione degli use case ultimata.

\subsection{Unadjusted Actor Weight}

Il \code{UAW} sar\`a determinato a individuazione degli use case ultimata.

\subsection{Technical Complexity Factor}

Il \code{TCF} \`e un fattore moltiplicativo che tiene conto della complessit\`a tecnica del sistema in sviluppo.
Ciascun fattore tecnico ha un peso compreso fra $0.5$ e $2.0$.
A ciascun fattore tecnico viene assegnato un valore fra $0$ (fattore ininfluente) e $5$ (fattore essenziale)
Il \code{TCF} viene calcolato sommando il prodotto di peso e valore per ciascun fattore tecnico.

I valori assegnati sono indicati nella tabella seguente.

\begin{center}
\begin{tabularx}{\columnwidth}{c X c c}
\toprule
\cellcolor{color2!10} Fatt. & \cellcolor{color2!10} Descrizione & \cellcolor{color2!10} Peso & \cellcolor{color2!10} Val. \\
\midrule
\code{T1} & Sistema distribuito & $2.0$ & --- \\
\code{T2} & Obiettivi di performance/tempo di risposta & $1.0$ & --- \\
\code{T3} & Efficienza lato utente & $1.0$ & --- \\
\code{T4} & Complessit\`a processi interni & $1.0$ & --- \\
\code{T5} & Riusabilit\`a del codice & $1.0$ & --- \\
\code{T6} & Facilit\`a d'installazione & $0.5$ & 3 \\
\code{T7} & Facilit\`a d'utilizzo & $0.5$ & 4 \\
\code{T8} & Portabilit\`a inter-piattaforma & $2.0$ & --- \\
\code{T9} & Manutenzione del sistema & $1.0$ & --- \\
\code{T10} & Parallelismo/concorrenza & $1.0$ & --- \\
\code{T11} & Sicurezza & $1.0$ & 5 \\
\code{T12} & Accesso terze parti & $1.0$ & 5 \\
\code{T13} & Addestramento utente & $1.0$ & 1 \\
\midrule
\multicolumn{3}{r}{Totale \code{TCF}} & --- \\
\bottomrule
\end{tabularx}
\end{center}

La scelta dei valori \`e stata basata sulle seguenti considerazioni:
\begin{itemize}
	\item \code{T6}: la facilit\`a d'installazione \`e un fattore mediamente rilevante.
		Il software \`e destinato a enti bancari, e l'installazione sar\`a curata da membri della nostra societ\`a.
	\item \code{T7}: la facilit\`a d'utilizzo \`e una componente molto importante del sistema.
		Il sistema in sviluppo riguarda questioni finanziarie della vita del singolo, per le quali la facilit\`a d'uso non \`e importante quanto altre questioni, principalmente la sicurezza del sistema e dei risparmi degli utenti: a titolo d'esempio, se operazioni sensibili come inviare un bonifico richiedono pi\`u step da completare con attenzione \`e ragionevole pensare che sia pi\`u difficile effettuare un bonifico per errore.
		La facilit\`a d'uso \`e pi\`u rilevante per i dipendenti della banca.
	\item \code{T11}: la sicurezza \`e uno degli aspetti fondamentali del progetto.
	\item \code{T12}: l'accesso a parti terze \`e uno degli aspetti fondamentali del progetto.
	\item \code{T13}: l'addestramento utente \`e praticamente irrilevante.
		Gli utenti finali del sistema di Home Banking (clienti della banca) non possono essere addestrati.
		I dipendenti della banca possono essere addestrati, ma la necessit\`a di addestramento dovrebbe essere mitigata dalla facilit\`a d'uso del software (fattore \code{T7}).
\end{itemize}

\subsection{Environmental Complexity Factor}

Il \code{ECF} \`e un fattore moltiplicativo che tiene conto della complessit\`a dell'ambiente in cui il software viene sviluppato.
Ciascun fattore ambientale ha un peso compreso fra $0.5$ e $2.0$.
A ciascun fattore ambientale viene assegnato un valore fra $0$ (fattore ininfluente) e $5$ (fattore essenziale)
Il \code{ECF} viene calcolato sommando il prodotto di peso e valore per ciascun fattore ambientale.

I valori assegnati sono indicati nella tabella seguente.

\begin{center}
\begin{tabularx}{\columnwidth}{c X r r}
\toprule
\cellcolor{color2!10} Fatt. & \cellcolor{color2!10} Descrizione & \cellcolor{color2!10} Peso & \cellcolor{color2!10} Val. \\
\midrule
\code{E1} & Familiarit\`a con processo software adottato & $1.5$ & --- \\
\code{E2} & Esperienza nel dominio dell'applicazione & $0.5$ & --- \\
\code{E3} & Esperienza del team con orientazione a oggetti & $1.0$ & --- \\
\code{E4} & Capacit\`a primo analista & $0.5$ & --- \\
\code{E5} & Motivazione del team & $1.0$ & --- \\
\code{E6} & Stabilit\`a dei requisiti & $2.0$ & --- \\
\code{E7} & Staff part-time & $-1.0$ & --- \\
\code{E8} & Linguaggio di programmazione ostico & $-1.0$ & --- \\
\midrule
\multicolumn{3}{r}{Totale \code{ECF}} & --- \\
\bottomrule
\end{tabularx}
\end{center}

\subsection{Use Case Points e previsione di costo}

Gli Use Case Points saranno determinati a individuazione degli use case ultimata.
