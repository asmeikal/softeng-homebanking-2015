\def\iducMNEMO{UC\_MNEMO\_1\,}
\def\shortiducMNEMO{UC\_1\,}
\def\iducBIDVIS{UC\_BIDVIS\_2\,}
\def\shortiducBIDVIS{UC\_2\,}
\def\iducDISPAG{UC\_DISPAG\_3\,}
\def\shortiducDISPAG{UC\_3\,}
\def\iducCROPVEL{UC\_CROPVEL\_4\,}
\def\shortiducCROPVEL{UC\_4\,}
\def\iducCLIACC{UC\_CLIACC\_5\,}
\def\shortiducCLIACC{UC\_5\,}
\def\iducUSRBID{UC\_USRBID\_6\,}
\def\shortiducUSRBID{UC\_6\,}
\def\iducCREABID{UC\_CREABID\_7\,}
\def\shortiducCREABID{UC\_7\,}
\def\iducMNEMO{UC\_MNEMO\_8\,}
\def\shortiducMNEMO{UC\_8\,}
\def\iducISCRCORR{UC\_ISCRCORR\_9\,}
\def\shortiducISCRCORR{UC\_9\,}
\def\iducAPPRCORR{UC\_APPRCORR\_10\,}
\def\shortiducAPPRCORR{UC\_10\,}
\def\iducDISOPVEL{UC\_DISOPVEL\_11\,}
\def\shortiducDISOPVEL{UC\_11\,}
\def\iducMNEMO{UC\_MNEMO\_12\,}
\def\shortiducMNEMO{UC\_12\,}


\subsection{\code{UC\_1} - Titolo }
\label{sec:use-case:MNEMO}

\begin{ptable}{2}
\ptitle{Titolo}
\pcell{1}{
	Titolo (\code{UC\_MNEMO\_1})
}
\pline
\ptitle{Descrizione use case}
\pcell{1}{
		description
}
\pline
\ptitle{Attori}
\pcell{1}{
		actors
}
\pline
\ptitle{Origine}
\pcell{1}{
		origin
}
\pline
\ptitle{Pre-condizioni}
\pcell{1}{
		preconditions
}
\pline
\ptitle{Flusso}
\pcell{1}{
		use case flow
}
\pline
\ptitle{Post-condizioni}
\pcell{1}{
		postconditions
}
\pline
\ptitle{Side effects}
\pcell{1}{
		side effects
}
\pline
\ptitle{Flusso alternativo}
\pcell{1}{
		alternative flow
}
\pline
\ptitle{Post-condizioni alternative}
\pcell{1}{
		alternative post conditions
}
\pline
\ptitle{Side effects alternativi}
\pcell{1}{
		alternative side effects
}
\end{ptable}

\subsection{\code{UC\_2} - Visualization Bidding }
\label{sec:use-case:BIDVIS}

\begin{ptable}{2}
\ptitle{Titolo}
\pcell{1}{
	Visualization Bidding (\code{UC\_BIDVIS\_2})
}
\pline
\ptitle{Descrizione use case}
\pcell{1}{
		I manager della banca possono ottenere delle \emph{visualization} delle regole di bidding realizzate. Una \emph{visualization} fornisce una rappresentaziona grafica intuitiva dell'insieme di bid approvati automaticamente, soggetti ad approvazione da parte di un manager, e respinti automaticamente.
}
\pline
\ptitle{Attori}
\pcell{1}{
		Manager della banca
}
\pline
\ptitle{Origine}
\pcell{1}{
		Requisiti funzionali e requisiti di usabilit\`a.
}
\pline
\ptitle{Pre-condizioni}
\pcell{1}{
		Almeno una regola di bidding \`e stata definita.
}
\pline
\ptitle{Flusso}
\pcell{1}{
		\begin{enumerate} \item Il manager seleziona una regola di bidding definita in HBS; \item il sistema di HBS realizza una \emph{visualization} della regola di bidding; \item la \emph{visualization} prodotta viene mostrata al manager nel browser. \end{enumerate}
}
\pline
\ptitle{Post-condizioni}
\pcell{1}{
		Nessuna
}
\pline
\ptitle{Side effects}
\pcell{1}{
		Nessuno
}
\end{ptable}

