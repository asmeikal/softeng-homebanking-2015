\section{Corrispondenza casi d'uso / requisiti}

\newcommand{\vertical}[1]{\rotatebox[origin=c]{90}{#1}}
\newcommand{\V}{\checkmark}

Nella tabella \ref{table:tracking-requisiti} \`e illustrata la corrispondenza fra casi d'uso e requisiti funzionali del sistema.
Una spunta (\V) nella casella in riga $i$, colonna $j$, indica che il caso d'uso in riga $i$ realizza il requisito funzionale in colonna $j$.

%Magia: posso riferirmi al requisito \ref{req:itm:utente:funzionali:iscrizione}.
%Posso anche stampare l'id di un requisito, ad esempio \idLOGOP (o \shortidLOGOP in breve).

\begin{table}[h]
\resizebox{1\textwidth}{!}{\begin{minipage}{\textwidth}
\begin{center}
\begin{tabular}{r*{8}{|c}}
	& \vertical{\idSECAUTH}
	& \vertical{\idISCRCORR}
	& \vertical{\idLOGOP}
	& \vertical{\idCROPVEL}
	& \vertical{\idDISOPVEL}
	& \vertical{\idDISPAG}
	& \vertical{\idVERTIT}
	& \vertical{\idVERSAL} \\ \hline
	Prova &  & \checkmark & & \checkmark & & & & \\
\end{tabular}
\end{center}
\caption[Corrispondenza requisiti]{Corrispondenza fra requisiti e casi d'uso.}
\label{table:tracking-requisiti}
\end{minipage} }
\end{table}
