
\section{Specifica dei requisiti}

Di seguito \`e illustrata la specifica dettagliata dei requisiti.

\subsection{Requisiti funzionali}

\emph{Da determinare}

\subsection{Requisiti non funzionali}

\emph{Da determinare}

\subsubsection{Requisiti di sicurezza del sistema}
 
Le credenziali d'accesso sono per un account sono:
\begin{itemize}
	\item dati anagrafici del correntista;
	\item numero conto corrente;
	\item password scelta dal correntista.
\end{itemize}
	
Una volta che ha effettuato l'accesso,  l'utente deve poter svolgere le seguenti operazioni senza fornire la One Time Password:
\begin{itemize}
	\item Visionare saldo contabile, disponibile e liquido.
	\item Visionare uno storico delle transazioni effettuate.
	\item Visionare informazioni riguardo le carte di credito collegate al conto (se presenti).
	\item Effettuare ``operazioni veloci'' impostate attraverso un sistema di configurazione.
\end{itemize}

Invece è necessario fornire la One Time Password per:
\begin{itemize}
	\item effettuare transazioni, come:
	\begin{itemize}
		\item bonifici ordinari e bonifici SEPA;
		\item ricariche carte prepatate e schede telefoniche;
		\item pagamento bollette, bollettini, tasse, etc;
	\end{itemize}
	\item configurare operazioni veloci;
	\item ogni altra operazione sensibile.
\end{itemize}

Ogni operazione effettuata da un utente sul sistema deve essere registrata in un log.
In particolare, ogni log deve contenere almeno:
\begin{itemize}
	\item l'operazione eseguita;
	\item il conto coinvolto nell'operazione;
		%TODO rivedere codesta parte: codice univoco della transazione al posto del conto?
		% analizzare bene come funziona OBP
	\item l'istante dell'operazione;
	\item informazioni riguardanti il terminale da cui \`e stata effettuata l'operazione.
\end{itemize}

\subsection{Requisiti di dominio}

La legislazione attuale richiede che gli organi di controllo finanziario come la Banca d'Italia e le forze dell'ordine possano accedere in lettura a tutte le informazioni salvate dal sistema di Online Banking.

Ogni sistema informatico nell'ambito bancario deve permettere l'accesso da remoto alla sua rete interna tramite il meccanismo di \emph{VPN}.
Fornire alle autorit\`a di controllo le credenziali e/o interfacce per accedere ai sistemi di \emph{data storage}.

In particolare le forze dell'ordine devono poter:
\begin{itemize}
    \item Dato un utente visualizzare le seguenti informazioni:
        \begin{itemize}
            \item Tutte le transazioni effettuati dall'utente.
            \item Tutte le transazioni che hanno l'utente come destinatario.
            \item Dati anagrafici dell'utente.
            \item Informazioni riguardo al terminale informatico dal quale l'\emph{account} dell'utente \`e stato acceduto in precedenza.
        \end{itemize}
    \item In caso le informazioni siano ridondanti il sistema pu\`o fornire alle ff. oo. un modo per confrontare le informazioni e fare il controllo di coerenza.
\end{itemize}
        %TODO da qualche parte bisogna dire che un haxxor in genere non puo modificare tutti i log in maniera coerente, perche pensa a rubare i dindi invece di giocare ad uplink

L'organo di controllo finanziario deve poter:
\begin{itemize}
    \item Dato un utente visualizzare le seguenti informazioni:
        \begin{itemize}
            \item Tutte le transazioni effettuati dall'utente.
            \item Tutte le transazioni che hanno l'utente come destinatario.
            \item Lo storico di Online Trading dell'utente.
        \end{itemize}
\end{itemize}
La normativa legale richiede che un sistema in grado di offrire funzionalit\`a di trading fornisca all'organo di controllo finanziario l'accesso allo storico delle operazioni, in particolare l'organo di controllo deve poter:
\begin{itemize}
    \item Visualizzare le informazioni dei pacchetti di Online Trading.
    \item Visualizzre lo storico del sistema di Online Trading.
    %non so se ci va visualizzare mutui e tassi
\end{itemize}

Inoltre dato che il sistema di Bidding deve rispettare le normative legali, per esempio le leggi sull'usura bancaria, l'ente responsabile del controllo finanziario deve poter:
\begin{itemize}
    \item Accedere allo storico del sistema Bidding.
    \item Visualizzare i parametri entro i quali le proposte vengono accettate automaticamente.
    \item Visualizzare informazioni delle proposte accettate dal management.
    \item Visualizzare proposte negate dal management.
\end{itemize}

%TODO conti sono observable, la gg.ff. puo definire trigger. .. . \ldots .. \ldots \ldots eas\ldots \ldots .. \ldots \ldots ..




