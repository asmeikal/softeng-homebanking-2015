
\section{Specifica dei requisiti}

Di seguito \`e illustrata la specifica dettagliata dei requisiti.

\subsection{Requisiti funzionali}

\emph{Da determinare}

\subsection{Requisiti non funzionali}

\emph{Da determinare}

\subsubsection{Requisiti di sicurezza del sistema}
 
Le credenziali d'accesso sono per un account sono:
\begin{itemize}
	\item dati anagrafici del correntista;
	\item numero conto corrente;
	\item password scelta dal correntista.
\end{itemize}
	
Una volta che ha effettuato l'accesso,  l'utente deve poter svolgere le seguenti operazioni senza fornire la One Time Password:
\begin{itemize}
	\item Visionare saldo contabile, disponibile e liquido.
	\item Visionare uno storico delle transazioni effettuate.
	\item Visionare informazioni riguardo le carte di credito collegate al conto (se presenti).
	\item Effettuare ``operazioni veloci'' impostate attraverso un sistema di configurazione.
\end{itemize}

Invece è necessario fornire la One Time Password per:
\begin{itemize}
	\item effettuare transazioni, come:
	\begin{itemize}
		\item bonifici ordinari e bonifici SEPA;
		\item ricariche carte prepatate e schede telefoniche;
		\item pagamento bollette, bollettini, tasse, etc;
	\end{itemize}
	\item configurare operazioni veloci;
	\item ogni altra operazione sensibile.
\end{itemize}

Ogni operazione effettuata da un utente sul sistema deve essere registrata in un log.
In particolare, ogni log deve contenere almeno:
\begin{itemize}
	\item l'operazione eseguita;
	\item il conto coinvolto nell'operazione;
		%TODO rivedere codesta parte: codice univoco della transazione al posto del conto?
		% analizzare bene come funziona OBP
	\item l'istante dell'operazione;
	\item informazioni riguardanti il terminale da cui \`e stata effettuata l'operazione.
\end{itemize}

\subsection{Requisiti di dominio}

\emph{Da determinare}
