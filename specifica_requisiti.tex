
\section{Specifica dei requisiti}

Di seguito \`e illustrata la specifica dettagliata dei requisiti.

\subsection{Requisiti funzionali}

\emph{Da determinare}

\subsection{Requisiti non funzionali}

\begin{itemize}
	\item Il sistema di Home Banking deve fornire ai suoi utenti (clienti della banca, dipendenti della banca, agenti di terze parti) la possibilit\`a di visualizzare uno storico dettagliato di ogni operazione (requisito funzionale) effettuata.
%TODO muovere fra i requisiti funzionali
%	Ogni operazione effettuata da un utente (cliente della banca o dipendente della banca) sul sistema deve essere registrata in un log.
%	Ogni log deve contenere quantomeno le seguenti informazioni:
%	\begin{itemize}
%		\item un'identificativo del tipo di operazione eseguita;
%		\item il conto coinvolto nell'operazione;
%		\item l'istante dell'operazione;
%		\item informazioni riguardanti il terminale da cui \`e stata effettuata l'operazione.
%	\end{itemize}

	\item Il sistema di Home Banking ha dei requisiti impliciti di disponibilit\`a, ovvero il sistema deve avere una percentuale minima garantita di \emph{uptime}, o tempo in cui il sistema \`e disponibile e utilizzabile dagli utenti.
	La percentuale \emph{uptime} non comprende eventuale tempo di \emph{downtime} previsto periodicamente per motivi di manutenzione, ottimizzazione delle prestazioni, aggiornamento, etc.
	Le percentuali di \emph{uptime} garantito e di \emph{downtime} richiesto devono essere definite in funzione della quantit\`a di operazioni gestita dal sistema, ossia in funzione del numero di utenti attivi presso un particolare istituto bancario e della mole di transazioni effettuate quotidianamente, e in funzione delle caratteristiche delle macchine su cui il software viene installato presso un particolare istituto bancario.
	La percentuale di \emph{uptime} garantito e di \emph{downtime} richiesto deve quindi essere stabilita e personalizzata al momento della vendita del software ad un istituto bancario.

	\item Il sistema di Home Banking ha dei requisiti impliciti di usabilit\`a, ovvero:
	\begin{itemize}
		\item deve essere utilizzabile senza particolare addestramento dai clienti della banca;
		\item deve essere utilizzabile dopo addestramento minimo dai dipendenti della banca.
	\end{itemize}
	Le operazioni disponibili per i clienti della banca (requisiti funzionali) devono essere facilmente utilizzabili e comprensibili.
	Ogni maschera per l'inserimento di informazioni deve essere corredata da opportuni testi brevi illustranti il tipo di informazioni richieste.
\end{itemize}

\subsubsection{Requisiti di sicurezza del sistema}

Il sistema di Home Banking presenta forti requisiti non funzionali di sicurezza.

\begin{itemize}
	\item L'accesso al sistema di Home Banking da parte di un utente avviene a seguito di autenticazione dello stesso.
		Le credenziali di accesso devono essere trasmesse dal browser dell'utente al sistema di Home Banking utilizzando una connessione sicura.
	\item Le credenziali d'accesso sono per un account sono:
		\begin{itemize}
			\item numero conto corrente;
			\item password fornita al correntista al momento della registrazione.
		\end{itemize}
	\item Le operazioni effettuabili dall'utente (requisiti funzionali) sono partizionate in operazioni che richiedono un'ulteriore autenticazione tramite One Time Password e operazioni che \emph{non} richiedono ulteriore autenticazione.
	Ogni operazione \`e eseguibile solo dopo che l'utente ha effettuato l'accesso al sistema di Home Banking.

	Le seguenti operazioni non richiedono ulteriore autenticazione:
	\begin{itemize}
		\item Visionare saldo contabile, disponibile e liquido.
		\item Visionare uno storico delle transazioni effettuate.
		\item Visionare informazioni riguardo le carte di credito collegate al conto (se presenti).
		\item Effettuare ``operazioni veloci'' impostate attraverso un sistema di configurazione.
	\end{itemize}
	
	Ogni altra operazione richiede autenticazione tramite One Time Password.
	In particolare \`e richiesta autenticazione tramite One Time Password per:
	\begin{itemize}
		\item effettuare transazioni, come:
		\begin{itemize}
			\item bonifici ordinari, bonifici SEPA;
			\item ricariche carte prepatate e schede telefoniche;
			\item pagamento bollette, bollettini, tasse, etc;
		\end{itemize}
		\item configurare operazioni veloci.
	\end{itemize}
\end{itemize}

\subsection{Requisiti di dominio}

\emph{Da determinare}
