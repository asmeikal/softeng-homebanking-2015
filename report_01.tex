
\documentclass[a4paper,11pt]{texmemo}
\usepackage[italian]{babel}

\newcommand{\code}[1]{\texttt{#1}}

\memonumber{1}
\memosubject{Note su incontro con capoprogetto}
\memodate{12/04/2016}

\begin{document}

\maketitle

Di seguito sono riportate le considerazioni emerse durante l'incontro del 12 Aprile.

Rivedere e definire con maggiore precisione milestone e attivit\`a del progetto.
In particolare:
\begin{itemize}
	\item (\code{REV\_01\_01}) milestone della fase di Construction: l'avvenuta verifica del funzionamento nell'ambiente di sviluppo del software prodotto;
	\item (\code{REV\_01\_02}) attivit\`a della fase di Transition: installazione presso i clienti.
		Ha una conseguenza sui fattori di complessit\`a tecnica nell'analisi dei costi: aumentare il valore assegnato alla facilit\`a di installazione.
\end{itemize}

Nell'analisi dei rischi:
\begin{itemize}
	\item (\code{REV\_01\_03}) rivedere il \emph{management} dei rischi \code{RIS\_03} e \code{RIS\_04};
	\item (\code{REV\_01\_04}) identificare i \emph{trigger} di ciascun rischio.
\end{itemize}

Nell'analisi del contesto tenere conto di:
\begin{itemize}
	\item (\code{REV\_01\_05}) bonifici all'interno della stessa banca;
	\item (\code{REV\_01\_06}) pagamento tramite MAV\cite{mav_wiki} e RAV.
\end{itemize}

Nella definizione dei requisiti:
\begin{itemize}
	\item (\code{REV\_01\_07}) nella sezione sul \emph{bidding} valutare la possibilit\`a di automatizzare la gestione di prestiti e mutui;
	\item (\code{REV\_01\_08}) valutare se inserire la compravendita di pacchetti di titoli preconfezionati dalla banca.
\end{itemize}

\renewcommand{\refname}{Riferimenti}
\printbibliography

\end{document}
