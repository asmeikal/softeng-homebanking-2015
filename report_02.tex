
\documentclass[a4paper,11pt]{texmemo}
\usepackage[italian]{babel}

\newcommand{\code}[1]{\texttt{#1}}

\memonumber{2}
\memosubject{Note su compravendita di pacchetti preconfezionati}
\memodate{15/04/2016}

\begin{document}

\maketitle

Nel memo \# 1 si propone l'inserimento di una funzionalit\`a di compravendita da parte degli utenti di HBS di pacchetti preconfezionati da dipendenti della banca (\code{REV\_01\_08}) per consentire una gestione (seppur parziale) da parte dell'utente di HBS del suo portafoglio fondi.

La funzionalit\`a proposta (informalmente chiamata \emph{Home Trading}) \`e descritta di seguito:
\begin{itemize}
	\item la banca che adotta HBS confeziona \emph{pacchetti di investimento} che vengono proposti agli utenti;
	\item l'utente pu\`o visualizzare lo storico delle azioni contenute in un pacchetto, una previsione di andamento delle stesse e una misura del loro rischio, e valutarne l'acquisto;
	\item l'acquisto di un pacchetto vincola l'utente al possesso dello stesso per un periodo di tempo pre-determinato;
	\item al termine del periodo l'utente riceve una somma commisurata al nuovo valore delle azioni.
\end{itemize}

\`E necessaria la consultazione con esperti del settore per stabilire i dettagli operativi della funzionalit\`a proposta.
In particolare \`e necessario determinare la valenza giuridica dei \emph{pacchetti di investimento} e quantificare il rischio massimo cui gli utenti di HBS possono essere esposti dalla banca.
\`E inoltre necessario stabilire il rapporto fra guadagno/perdita dei titoli azionari, e fra il guadagno/perdita della banca e degli utenti di HBS.

\end{document}
