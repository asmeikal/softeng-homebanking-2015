
\documentclass[a4paper,11pt]{texmemo}
\usepackage[italian]{babel}

\newcommand{\code}[1]{\texttt{#1}}

\memonumber{3}
\memosubject{Note su incontro con capoprogetto}
\memodate{10/05/2016}

\begin{document}

\maketitle

Di seguito sono riportare le considerazioni emerse durante l'incontro con il capoprogetto durante la seconda iterazione della fase di inception.

\begin{itemize}
	\item (\code{REV\_03\_01}) correggere divisione requisito di logging fra requisiti funzionali e requisiti funzionali.
	\item (\code{REV\_03\_02}) aggiugere \emph{Home Trading} a diagramma degli use case.
	\item (\code{REV\_03\_03}) specificare nella proposta di progetto l'idea di business.

		In particolare il prodotto che la nostra societ\`a realizza \`e un software di Home Banking che utilizza l'interfaccia OBP per la comunicazione con il back-end della banca.

		La vendita del prodotto avviene assieme all'installazione dello stesso da parte della nostra societ\`a presso l'acquirente.

		Il software realizzato ha come unico requisito la presenza di una piattaforma di virtualizzazione sulle macchine dell'acquirente.
		In assenza del software e/o delle macchine necessarie la nostra societ\`a si impegna a fornire l'hardware necessario predisposto all'installazione a costo aggiuntivo.

		Illustrare procedura di vendita e installazione tramite un business model appropriato.
	\item (\code{REV\_03\_04}) aggiungere al glossario:
		\begin{itemize}
			\item microservizi
			\item macchine virtuali
		\end{itemize}
\end{itemize}

\end{document}
