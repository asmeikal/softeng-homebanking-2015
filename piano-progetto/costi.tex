\section{Analisi dei costi}

La stima dei costi di realizzazione del progetto viene effettuata con la tecnica degli \emph{Use Case Points}.

Gli Use Case Points del progetto sono dati dalla somma di due addendi.

\paragraph{Unadjusted Use Case Weight}
	Misura il numero e la complessit\`a degli Use Case del sistema.

\paragraph{Unadjusted Actor Weight}
	Misura il numero e la complessit\`a degli attori del sistema.

Il valore ottenuto dalla somma di \code{UUCW} e \code{UAW} viene moltiplicato per due fattori.

\paragraph{Technical Complexity Factor}
	Misura la complessit\`a tecnica del sistema.

\paragraph{Environmental Complexity Factor}
	Misura la complessit\`a dell'ambiente in cui il sistema viene sviluppato.

\subsection{Unadjusted Use Case Weight}

Il \code{UUCW} assegna un punteggio a ciascun use case del sistema secondo la complessit\`a dello use case.
Uno use case viene classificato secondo uno di tre valori, a seconda del numero di transazioni necessarie a portarlo a termine:
\begin{description}
	\item[Semplice] da 1 a 3 transazioni;
	\item[Medio] da 4 a 7 transazioni;
	\item[Complesso] da 8 transazioni in su.
\end{description}
Alle tre categorie viene assegnato, rispettivamente, un peso di 5, 10, e 15.
Il \code{UUCW} viene calcolato sommando il peso di tutti gli use case del sistema.

Nella tabella seguente sono illustrati gli use case del sistema di HBS e i relativi pesi.

\begin{center}
\begin{tabularx}{\columnwidth}{X l l}
\toprule
\cellcolor{color2!10} Use case & \cellcolor{color2!10} Classificazione & \cellcolor{color2!10} Peso \\
\midrule
\iducVERSTOR & Medio & 10 \\
\iducBIDVIS & Complesso & 15 \\
\iducDISPAG & Complesso & 15 \\
\iducCROPVEL & Complesso & 15 \\
\iducCLIACC & Medio & 10 \\
\iducUSRBID & Complesso & 15 \\
\iducCREABID & Complesso & 15 \\
\iducISCRCORR & Complesso & 15 \\
\iducAPPRCORR & Medio & 10 \\
\iducDISOPVEL & Complesso & 15 \\
\iducVERSAL & Medio & 10 \\
\midrule
\multicolumn{2}{r}{Totale \code{UUCW}} & 145 \\
\bottomrule
\end{tabularx}
\end{center}

\subsection{Unadjusted Actor Weight}

Il \code{UAW} viene calcolato in base al numero e alla complessit\`a degli attori del sistema.
Un attore viene classificato secondo uno di tre valori, a seconda della sua complessit\`a:
\begin{description}
	\item[Semplice] un attore esterno che interagisce con il sistema utilizzando una API ben definita;
	\item[Medio] un attore esterno che interagisce con il sistema utilizzando un protocollo di comunicazione standardizzato;
	\item[Complesso] un attore umano che interagisce con il sistema utilizzando un'interfaccia grafica.
\end{description}
Alle tre categorie viene assegnato, rispettivamente, un peso di 1, 2, e 3.
Il \code{UAW} viene calcoalto sommando il peso di tutti gli attori del sistema.

Gli attori di HBS e relative categorizzazioni sono illustrati nella tabella seguente.
Oltre agli attori definiti nel modello degli use case sono indicati anche i sistemi esterni che dovranno interagire con il sistema.

\begin{center}
\begin{tabularx}{\columnwidth}{X l l}
\toprule
\cellcolor{color2!10} Attore & \cellcolor{color2!10} Classificazione & \cellcolor{color2!10} Peso \\
\midrule
Cliente di HBS & Complesso & 3 \\
Dipendente della banca & Complesso & 3 \\
Manager della banca & Complesso & 3 \\
Ente di controllo & Complesso & 3 \\
OBP & Medio & 2 \\
OTP & Medio & 2 \\
\midrule
\multicolumn{2}{r}{Totale \code{UAW}} & 16 \\
\bottomrule
\end{tabularx}
\end{center}

\subsection{Technical Complexity Factor}

Il \code{TCF} \`e un fattore moltiplicativo che tiene conto della complessit\`a tecnica del sistema in sviluppo.
Ciascun fattore tecnico ha un peso compreso fra $0.5$ e $2.0$.
A ciascun fattore tecnico viene assegnato un valore fra $0$ (fattore ininfluente) e $5$ (fattore essenziale)
Il \code{TCF} viene calcolato sommando il prodotto di peso e valore per ciascun fattore tecnico, dividendo per 100 e aggiungendo $0.6$.

I valori assegnati sono indicati nella tabella seguente.

\begin{center}
\begin{tabularx}{\columnwidth}{c X c c}
\toprule
\cellcolor{color2!10} Fatt. & \cellcolor{color2!10} Descrizione & \cellcolor{color2!10} Peso & \cellcolor{color2!10} Val. \\
\midrule
\code{T1} & Sistema distribuito & $2.0$ & 4 \\
\code{T2} & Obiettivi di performance/tempo di risposta & $1.0$ & 2 \\
\code{T3} & Efficienza lato utente & $1.0$ & 3 \\
\code{T4} & Complessit\`a processi interni & $1.0$ & 4 \\
\code{T5} & Riusabilit\`a del codice & $1.0$ & 3 \\
\code{T6} & Facilit\`a d'installazione & $0.5$ & 5 \\
\code{T7} & Facilit\`a d'utilizzo & $0.5$ & 4 \\
\code{T8} & Portabilit\`a inter-piattaforma & $2.0$ & 5 \\
\code{T9} & Manutenzione del sistema & $1.0$ & 5 \\
\code{T10} & Parallelismo/concorrenza & $1.0$ & 5 \\
\code{T11} & Sicurezza & $1.0$ & 5 \\
\code{T12} & Accesso terze parti & $1.0$ & 5 \\
\code{T13} & Addestramento utente & $1.0$ & 1 \\
\midrule
\multicolumn{3}{r}{Totale \code{TF}} & 50 \\
\bottomrule
\end{tabularx}
%\caption{\label{tab:tcf} Fattori di complessit\`a tecnica.}
\end{center}

Il \code{TCF} \`e quindi
\[
	0.6 + \frac{50}{100} = 0.6 + 0.5 = 1.1
\]

L'assegnamento dei valori \`e basato sulle seguenti considerazioni:
\begin{itemize}
	\item \code{T1}: il sistema sar\`a verosimilmente composto da pi\`u componenti distribuite, identificate dalle loro responsabilit\`a.
	\item \code{T2}: il sistema non ha particolari necessit\`a in termini di performance, le operazioni svolte non hanno particolare complessit\`a.
		Il tempo di risposta del sistema devono rientrare nei tempi di risposta di un generico sito web, dell'ordine dei pochi secondi, ed \`e quindi un fattore mediamente rilevante.
	\item \code{T3}: l'efficienza del sistema per i clienti \`e un fattore poco rilevante rispetto a altri fattori come la sicurezza del sistema (\code{T11}), è più rilevante per i dipendenti della banca. Indichiamo una media fra le due valutazioni.
	\item \code{T4}: i processi interni del sistema \`e presumibile siano molto complessi.
	\item \code{T5}: la riusabilit\`a del codice prodotto \`e un fattore rilevante del progetto.
		\`E indirettamente correlato alla manutebilit\`a del sistema, e direttamente correlato alla possibilit\`a di implementare \emph{plugin} per il sistema.
	\item \code{T6}: la facilit\`a d'installazione \`e un fattore essenziale.
		Il software \`e destinato a enti bancari, e l'installazione sar\`a curata da membri della nostra societ\`a come parte preponderante della fase di Transition.
	\item \code{T7}: la facilit\`a d'utilizzo \`e una componente molto importante del sistema.
		Il sistema in sviluppo riguarda questioni finanziarie della vita del singolo, per le quali la facilit\`a d'uso non \`e importante quanto altre questioni, principalmente la sicurezza del sistema e dei risparmi degli utenti: a titolo d'esempio, se operazioni sensibili come inviare un bonifico richiedono pi\`u step da completare con attenzione \`e ragionevole pensare che sia pi\`u difficile effettuare un bonifico per errore.
		La facilit\`a d'uso \`e pi\`u rilevante per i dipendenti della banca.
	\item \code{T8}: la portabilit\`a del software sviluppato fra pi\`u piattaforme \`e un fattore essenziale del progetto, poich\'e non si possono avere informazioni precise in anticipo sulle macchine utilizzate da chi acquister\`a il nostro software.
	\item \code{T9}: la manutenzione del sistema \`e una delle parti fondamentali della fase di transition, ed \`e quindi un fattore essenziale nello sviluppo dello stesso.
	\item \code{T10}: le operazioni che dovr\`a svolgere il sistema di home banking in sviluppo avranno sicuramente degli aspetti di gestione della concorrenza, ad esempio nella gestione delle operazioni di un utente.
	\item \code{T11}: la sicurezza \`e uno degli aspetti fondamentali del progetto in quanto requisito non funzionale.
	\item \code{T12}: l'accesso a parti terze \`e uno degli aspetti fondamentali del progetto in quanto requisito di dominio.
	\item \code{T13}: l'addestramento utente \`e praticamente irrilevante.
		Gli utenti finali del sistema di Home Banking (clienti della banca) non possono essere addestrati.
		I dipendenti della banca possono essere addestrati, ma la necessit\`a di addestramento dovrebbe essere mitigata dalla facilit\`a d'uso del software (fattore \code{T7}).
\end{itemize}

\subsection{Environmental Complexity Factor}

Il \code{ECF} \`e un fattore moltiplicativo che tiene conto della complessit\`a dell'ambiente in cui il software viene sviluppato.
Ciascun fattore ambientale ha un peso compreso fra $0.5$ e $2.0$.
A ciascun fattore ambientale viene assegnato un valore fra $0$ (fattore ininfluente) e $5$ (fattore essenziale)
Il \code{ECF} viene calcolato sommando il prodotto di peso e valore per ciascun fattore ambientale, moltiplicandolo per $-0.03$ e sommandovi $1.4$.

I valori assegnati sono indicati nella tabella seguente.
%I valori assegnati sono indicati nella tabella \ref{tab:ecf}.

\begin{center}
\begin{tabularx}{\columnwidth}{c X r r}
\toprule
\cellcolor{color2!10} Fatt. & \cellcolor{color2!10} Descrizione & \cellcolor{color2!10} Peso & \cellcolor{color2!10} Val. \\
\midrule
\code{E1} & Familiarit\`a con processo software adottato & $1.5$ & 3 \\
\code{E2} & Esperienza nel dominio dell'applicazione & $0.5$ & 2 \\
\code{E3} & Esperienza del team con orientazione a oggetti & $1.0$ & 4 \\
\code{E4} & Capacit\`a primo analista & $0.5$ & 3 \\
\code{E5} & Motivazione del team & $1.0$ & 3 \\
\code{E6} & Stabilit\`a dei requisiti & $2.0$ & 4 \\
\code{E7} & Staff part-time & $-1.0$ & 0 \\
\code{E8} & Linguaggio di programmazione ostico & $-1.0$ & 3 \\
\midrule
\multicolumn{3}{r}{Totale \code{EF}} & 20 \\
\bottomrule
\end{tabularx}
%\caption{\label{tab:ecf} Fattori di complessit\`a ambientale.}
\end{center}

Il \code{ECF} \`e quindi
\[
	1.4 - 0.03 \cdot 20 = 1.4 - 0.6 = 0.8
\]

L'assegnamento dei valori \`e basato sul fatto che il software in sviluppo \`e \emph{sensibile}, in quanto concerne transazioni economiche di privati e presenta quindi problematiche importanti legate a sicurezza, aspetti legali, e privacy degli utenti.
Il progetto richiede quindi un team competente ed esperto tanto nel dominio quanto in aspetti legati al processo adottato e in generale alle pratiche di sviluppo \emph{object oriented}.
Per queste stesse considerazioni, e per i tempi di sviluppo condizionati dall'evoluzione dello scenario legale (relativo al dominio bancario) italiano, la stabilit\`a dei requisiti \`e un fattore molto rilevante.

\subsection{Use Case Points e previsione di costo}

Gli \code{UCP} sono calcolati sommando \code{UUCW} e \code{UAW}, e moltiplicando il valore ottenuto per i due fattori di complessit\`a, \code{TCF} e \code{ECF}.
Gli \code{UCP} di HBS sono quindi
\[
	(145 + 16) \cdot 0.8 \cdot 1.1 = 106.26
\]

Gli \code{UCP} possono essere convertiti in una misura di sforzo espressa in ore/uomo o in mesi/uomo assegnando una stima di ore/uomo necessarie per Use Case Point.
Supponendo di dover impiegare 30 ore/uomo per \code{UCP}, otteniamo
\[
	106.26 \cdot 30 = 3187.80 \text{ ore/uomo}
\]
che equivalgono, in mesi/uomo (assumendo 8 ore lavorative giornaliere per una media di 22 giorni lavorativi mensili), a
\[
	\frac{3187.80}{8 \cdot 22} = 18.11 \text{ mesi/uomo}
\]
La stima del costo del progetto di HBS, ipotizzando un costo medio di 5,000 \EUR/mese, \`e di circa 90,000 \EUR.

