\documentclass[]{softeng}

\usepackage[backend=bibtex]{biblatex}
\bibliography{softeng}

\title{Analisi del contesto}
\date{\today}

\begin{document}

\maketitle

Registrazione: come

Chiamiamo ``utente registrato'' un utente che possiede un account sul sito di home banking di una particolare banca, e che quindi abbia fornito informazioni anagrafiche e di contatto, e opzionalmente informazioni sulla sua condizione economica.

Un utente registrato pu\`o ottenere una lista delle tipologie di servizi offerti dalla banca che pu\`o sottoscrivere.

Un utente pu\`o aprire diversi tipi di conto corrente, come 

Un utente titolare di un conto.

Audit di sicurezza:
storico delle connessioni e delle operazioni riguardanti il conto corrente.

Il sistema di home banking \`e personalizzabile dagli impiegati della banca.
I dipendenti della banca possono creare:
- conti di deposito
- carte di credito/debito
Personalizzando vari parametri.
I dirigenti possono permettere l'apertura di certi conti/carte solo a clienti che abbiano i requisiti specificati:
- giacenza media
- capitale depositato che aumenta
- boh

Possibilit\`a di bidding in cui un utente fa una proposta alla banca per ottenere un conto o una carta con certe particolari condizioni/agevolazioni.
In base a delle regole definite dai dirigenti questa proposta pu\`o essere:
- approvata automaticamente dal sistema
- inoltrata a un manager per l'approvazione
- rifiutata automaticamente
Controllare se la cosa \`e legale o se ce ne freghiamo.

\section{Portafoglio fondi}




\end{document}