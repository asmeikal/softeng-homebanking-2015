%----------------------------------------------------------------------------------------
% Proposta di progetto
%----------------------------------------------------------------------------------------

\documentclass[10pt]{softeng} % Document font size and equations flushed left

%----------------------------------------------------------------------------------------
%	DOCUMENT INFORMATION
%----------------------------------------------------------------------------------------

\Phase{Inception - I iterazione}

\DocumentTitle{Requisiti di progetto} % Document title

%----------------------------------------------------------------------------------------

\begin{document}

\startofdocument{}

\section{Glossario}
%(versione 1 : sono omesse definizioni banali e/o implicite)

\paragraph{Audit di sicurezza} 
	operazioni di controllo del sistema effettuate periodicamente per controllare che non ci siano sate violazioni della politica di sicurezza adottata
\paragraph{Banca D'Italia}
	 Banca centrale della Repubblica Italiana, avente funzione sia di vigilanza su banche, istituti di credito, intermediari finanziari sia di principale controllore in materia di antiriciclaggio: insieme alla CONSOB, deve avere libero accesso, entro i termini previsti dalla legge, ai dati contenuti nei database di qualsiasi istituto finanziario che operi sul territorio italiano. \cite{banca_italia}
\paragraph{Bonifico}     
	operazione bancaria mediante la quale si mette a disposizione di una persona o le si accredita una somma di denaro per ordine e conto di altri
\paragraph{Bonifico Sepa}
	bonifico (v.bonifico) accreditabile ad un destinatario ubicato oltre i confini nazionali nell'area SEPA.
\paragraph{Conto Corrente (cc)}
	indica un deposito di danaro effettuato dal possesore del bene, detto \emph{correntista}, in una banca o istituto di credito
\paragraph{Codice IBAN}
	L'International Bank Account Number è uno standard internazionale utilizzato per identificare un'utenza bancaria e/o operazione associata ad un conto corrente 
\paragraph{Codice BIC/SWIFT}
	standard che definisce i \emph{bank identifier codes} (codici d'identificazione bancaria) approvato dall'International Organization for Standardization (ISO). Questi codici vengono utilizzati per i trasferimenti di denaro tra banche, specialmente nelle transazioni internazionali, per le quali è spesso ancora necessario nonostante l'entrata in vigore dell'IBAN. \cite{bic_wiki}
\paragraph{Commissione Nazionale per le societ\`a e la borsa (CONSOB)}
	è un'autorità amministrativa indipendente, dotata di personalità giuridica e piena autonomia la cui attività è rivolta alla tutela degli investitori, all'efficienza, alla trasparenza e allo sviluppo del mercato mobiliare italiano. \cite{consob_wiki}
\paragraph{Fondo comune di investimento}
	è un istituto d'intermediazione finanziaria mediante il quale è possibile partecipare, investita una determinata quota di danaro, alla gestione e alla spartizione di dividendi prodotti da un determinato \emph{bene mobiliare} nel tempo. La \emph{banca depositaria} ne custodisce materialmente i titoli e ne tiene in cassa le disponibilità liquide. Le banche hanno inoltre un ruolo di controllo sulla legittimità delle attività del fondo sulla base di quanto prescritto dalle norme della Banca d'Italia e dal regolamento del fondo stesso
\paragraph{Istituto di credito}
	organismo che svolge simultaneamente l’attività di raccolta di risorse finanziarie e di concessione del credito per proprio contoa terzi 
\paragraph{Open Bank Project (OBP)}
	è una API open source che permette a banche ed istituti di credito di creare un interfaccia utente di ampia portata e fruibilità. Essendo un sistema molto versatile e largamente riadattabile, si presta molto a definire un vero e proprio \emph{standard di interfaccia}, ossia a definire un canone per la creazione di interfacce rivolte e all'utenza bancaria generica e agli organismi deputati al controllo bancario. \cite{obp}
\paragraph{Portafoglio valori/titoli azionari}
	è l'insieme dei diversi titoli finanziari e/o fondi d'investimento che l'utente bancario generico può possedere.Ogni titolo e/o fondo acquisito viene inserito nel portafoglio.
\paragraph{Refactoring}
	processo di riutilizzazzione di codice già scritto in precedenza, senza doverne generare di nuovo
\paragraph{SEPA}
	La SEPA (Single Euro Payments Area) è l’area unica in cui i cittadini, le imprese e gli enti, possono eseguire e ricevere pagamenti in Euro, all’interno dei confini nazionali e tra i paesi diversi che compongono l’area SEPA con condizioni di base, diritti ed obblighi uniformi tra i paesi stessi. 
\paragraph{Time-based One Time Password (TOTP)}
	è un algoritmo che calcola una \emph{One-Time password} combinando mediante una funzione hash una chiave segreta condivisa ed il tempo corrente. \cite{totprfc}
\paragraph{Trading online}
	pratica uguale a quella del \emph{trading} bancario classico mediante aiuto di personalità con competenze specifiche(v.broker), effettuata però in rete, disponendo cioè di opportuni strumenti software per il monitoraggio di mercati azionari nazionali e internazionali e  per il controllo completo e \emph{real time} del proprio portafoglio azionario.


\section{Ripulire...}


Un utente registrato pu\`o ottenere una lista delle tipologie di servizi offerti dalla banca che pu\`o sottoscrivere.

Un utente pu\`o aprire diversi tipi di conto corrente, 

Un utente titolare di un conto.

Audit di sicurezza:
storico delle connessioni e delle operazioni riguardanti il conto corrente.

Il sistema di home banking \`e personalizzabile dagli impiegati della banca.
I dipendenti della banca possono creare:
- conti di deposito
- carte di credito/debito
Personalizzando vari parametri.
I dirigenti possono permettere l'apertura di certi conti/carte solo a clienti che abbiano i requisiti specificati:
- giacenza media
- capitale depositato che aumenta
- boh

Possibilit\`a di bidding in cui un utente fa una proposta alla banca per ottenere un conto o una carta con certe particolari condizioni/agevolazioni.
In base a delle regole definite dai dirigenti questa proposta pu\`o essere:
- approvata automaticamente dal sistema
- inoltrata a un manager per l'approvazione
- rifiutata automaticamente
Controllare se la cosa \`e legale o se ce ne freghiamo.



%--------------------------------------------------------------

% TODO ripulire questa parte e spostare tutto nel file dei requisiti
Per rafforzare i requisiti non funzionali (come i requisiti di sicurezza) si sceglie di rivolgere il sistema di Home Banking a persone fisiche (retail banking).

Alcune delle funzionalit\`a che il sistema pu\`o offrire sono:
\begin{enumerate}
	\item L'utente si deve poter pre-registrare online fornendo i suoi dati anagrafici, e:
		\begin{enumerate}
			\item se l'utente ha gi\`a un conto aperto con la banca, pu\`o fornire il suo numero di conto;
			\item se l'utente non ha gi\`a un conto con la banca, pu\`o scegliere diverse soluzioni con agevolazioni differenti e pre-compilare i moduli necessari all'apertura del conto.
				% TODO: cambiare "se l'utente ha gi\`a un conto con la banca" in "se l'utente desidera aprire un nuovo conto"
		\end{enumerate}
		Un utente pre-registrato pu\`o recarsi presso una filiale della banca e ultimare la registrazione presentando il documento d'identit\`a.
		La banca gli fornir\`a le informazioni per l'accesso.
		Ai nuovi clienti viene fornito su richiesta un bancomat.
		% TODO: rimuovere, non \`e di interesse per l'applicazione, facciamo home banking non ATM

		Un utente non pre-registrato pu\`o effettuare la procedura completa di registrazione presso una filiale, fornendo le stesse informazioni richieste agli utenti che si pre-registrano online.
	\item Le informazioni per l'accesso comprendono:
		\begin{enumerate}
			\item Il numero di conto;
			\item La password del conto;
			\item Un dispositivo TOTP\footnote{Time-Based One Time Password, RFC 6238. \url{https://tools.ietf.org/html/rfc6238}} per effettuare operazioni sensibili.
		\end{enumerate}
	\item Un utente registrato pu\`o effettuare il login al sistema di Home Banking fornendo le informazioni di accesso (requisito TOT), in particolare deve inserire numero di conto e password del conto.
	\item Un utente che ha effettuato l'accesso al sistema di Home Banking deve poter svolgere le seguenti operazioni senza fornire la One Time Password:
		\begin{enumerate}
			\item Visionare saldo contabile, disponibile e liquido.
			\item Visionare uno storico delle transazioni effettuate.
			\item Visionare informazioni riguardo le carte di credito collegate al conto (se presenti).
			\item Effettuare ``operazioni veloci'' impostate attraverso un sistema di configurazione (requisito TOT).
		\end{enumerate}
	\item Un utente che ha effettuato l'accesso al sistema di Home Banking deve poter effettuare le seguenti operazioni fornendo ogni volta la One Time Password:
		\begin{enumerate}
			\item Effettuare transazioni, come:
				\begin{enumerate}
					\item bonifici ordinari e bonifici SEPA;
					\item ricariche carte prepatate e schede telefoniche;
					\item pagamento bollette, bollettini, tasse, etc;
				\end{enumerate}
			\item Configurare operazioni veloci.
			\item Se ci viene in mente altro ce lo mettiamo.
		\end{enumerate}
	\item Ogni operazione deve essere registrata in un log.
		Devono essere mantenute le seguenti informazioni:
		\begin{enumerate}
			\item l'operazione eseguita;
			\item il conto coinvolto nell'operazione;
				% TODO rivedere codesta parte: codice univoco della transazione al posto del conto?
				% analizzare bene come funziona OBP
			\item l'istante dell'operazione;
			\item informazioni riguardanti il terminale da cui \`e stata effettuata l'operazione.
		\end{enumerate}
	\item Il sistema deve poter essere configurabile dall'utente per inviare notifiche via SMS e/o via email a seguito di ogni evento stabilito dall'utente.
		La banca pu\`o associare una tariffa per abilitare le notifiche su specifici insiemi di eventi.
		Gli eventi notificabili possono includere:
		\begin{enumerate}
			\item Accesso al sistema;
			\item Pagamento o prelievo dal conto;
			\item Pagamento ricevuto.
		\end{enumerate}
\end{enumerate}


%----------------------------------------------------------------------------------------
%	REFERENCE LIST
%----------------------------------------------------------------------------------------

\nocite{banca_italia}
\printcustombib{}

\end{document}
