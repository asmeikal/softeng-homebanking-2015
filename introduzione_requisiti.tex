\section{Introduzione}

Si vuole realizzare un sistema di Home Banking (da ora HBS) per la gestione di fondi privati a breve, medio e lungo termine via Web. Il sistema è rivolto a Banche che decidono di implementare il protocollo/software Open Bank Project. Si presuppone che la banca già abbia uno stabile ed efficiente sistema di database di proprietà con opportuni software accessori, e in generale non si vuole analizzare la parte relativa al back-end della banca.  

Per rispettare e meglio implementare alcuni fondamentali requisiti non funzionali (sicurezza, tempi di reazione del sistema, ecc.), durante la stesura della proposta di progetto si è deciso di rimanere nell'ambito del \emph{retail banking}, ossia di progettare un sistema i cui unici utenti siano persone fisiche e non imprese o altri istituti finanziari.

Ogni individuo iscritto ad una banca possiede almeno un conto su cui depositare i propri risparmi: può essere \emph{di deposito} o \emph{corrente} a seconda che siano permesse le sole operazioni di prelievo e deposito o che in aggiunta a queste sia possibile  effettuare bonifici, pagamento di assegni, prelievi bancomat e simili.

\subsection{Utenti del sistema}

Gli utenti di questo sistema si dividono principalmente in due categorie: utenti registrati, o correntisti, e dipendenti della banca. 

Un individuo può ottenere un \emph{account HBS} in due passi:
\begin{enumerate}
	\item
	\begin{enumerate}
		\item se è già correntista della banca, gli basta fornire ad un dipendente della banca il proprio numero di conto;
		\item se non è correntista, deve compilare i moduli necessari all'apertura di un conto corrente (on-line o presso una filiale);
	\end{enumerate}
	\item in ogni caso, l'individuo deve recarsi in una filiale per consegnare copia di documento d'identità e ricevere le 					\emph{credenziali d'accesso} del proprio account.
\end{enumerate}

%TODO ?e se un correntista vuole aprire un altro conto?

Le funzionalità che verranno descritte nel seguito per gli \emph{account di servizio} e per gli \emph{account HBS} sono un nucleo identificato come necessario per soddisfare le necessit\`a di un generico sistema di Home Banking.
%Il \emph{pool} di progettisti e sviluppatori si impegna a implementare questo kernel nel modo più riutilizzabile e scalabile possibile, in modo da lasciare alla banca che adotti questo sistema la possibilità di operare come meglio crede con queste funzionalità, ad esempio creandone altre a partire da quelle qui elencate.

Un correntista deve poter:
\begin{itemize} 
	\item visionare \emph{saldo contabile annuale}, relativo all'anno in corso con eventuali grafici Valore/Tempo dell'andamento del valore rispetto agli anni precedenti;
	\item visionare \emph{saldo contabile mensile}, relativo al mese corrente con eventuali grafici Valore/Tempo dell'andamento del valore rispetto ai mesi precedenti;
	\item visionare \emph{saldo liquido} relativo al giorno corrente, nel caso in cui voglia compiere operazioni bancarie che coinvolgano calcolo di \emph{interesse};
	\item visionare \emph{saldo disponibile}, relativo al giorno corrente;
	\item visionare uno storico delle transazioni effettuate, avendo a disposizione per ogni transizione la \emph{data contabile}, l'\emph{importo versato} ed una \emph{causale};
	\item gestire operazioni periodiche o programmate;
	\item visionare informazioni delle carte di credito eventualmente collegate al conto;
	\item effettuare operazioni ``veloci'', inserendo pochi dati in una maschera preconfigurata e confermando l'operazione mediante meccanismo TOTP:
	\begin{itemize}
			\item ricariche telefoniche;
%			\item bonifici ordinari e bonifici SEPA mediante compilazione di opportuni moduli on-line;
			\item pagamento delle bollette;
	\end{itemize}
	\item visionare l'andamento in borsa dei titoli azionari del suo portafoglio;
	\item vendere uno o più titoli azionari del proprio portafoglio;
	\item investire il proprio patrimonio nei titoli di credito statali (NON azionari) a medio e lungo termine che ritiene più opportuni, avendo a disposizione un \emph{benchmark} per valutarne l'andamento in borsa e il profilo di rischio.
\end{itemize}	

Ad ogni account HBS è associato, oltre al conto corrente, un portafoglio azionario per conservare tutti i titoli bancari di cui il correntista sia in possesso.
Un portafoglio azionario ha un valore, calcolato in funzione dei valori di ogni azione contenuta al suo interno: modifiche di questo valore influiscono sul conto corrente associato al portafoglio, e il correntista non solo deve poter visionare tale valore e il suo andamento nel tempo, ma deve anche poter gestire il contenuto del portafoglio stesso.

I dipendenti della banca si dividono in \emph{impiegati} e \emph{dirigenti}. A ogni dipendente, sia esso impiegato o dirigente, viene assegnato un \emph{account di servizio}, con cui egli può assolvere alle sue mansioni specifiche; i vari account di servizio differiscono per permessi di accesso e funzionalità disponibili.

Ad ogni \emph{account di servizio} sono assegnati:
\begin{itemize}
	\item i dati del relativo dipendente;
	\item una password ``speciale'' (decidere politica di sicurezza).
\end{itemize}	

Gli \emph{impiegati}, mediante il loro account di servizio, devono poter: 
\begin{itemize}
	\item confermare/respingere atti che richiedono esplicita approvazione, ad esempio i bid per conti o carte inviati dagli utenti; 
	\item effettuare eventuali operazioni minori di contabilità e di gestione che una banca potrebbe permettere via internet.
\end{itemize}

I \emph{dirigenti} devono poter:
\begin{itemize}
	\item accedere e modificare i disclaimer pubblicitari del sito di home banking;
	\item impostare e modificare opportuni pacchetti di azioni o di fondi d'investimento e in generale offerte che si vogliono propinare agli utenti;
	\item selezionare, mediante opportuna combinazione di \emph{queries}, categorie o fasce di utenti in base a diversi parametri socio-economici, ed ottenere precise statistiche al riguardo;
	\item confermare atti di importanza superiore (a.e. alti investimenti in titoli azionari, transazioni di danaro molto elevate da un conto corrente ad un altro, ecc.), in seguito all'approvazione iniziale di un impiegato.
\end{itemize}

\subsection{Transazioni}

Una transazione economica è il passaggio sicuro e irreversibile da un conto ad un altro di una certa quantità di denaro.
La somma viene accreditata al \emph{conto destinatario} e detratta al \emph{conto mittente}, indipendentemente dalle banche di appartenenza, sfruttando ``indirizzi bancari'' come il codice IBAN.

%TODO cambiare in "bonifici programmati"
Sebbene non sia possibile annullare l'effetto di una transazione, l'utente pu\`o usufruire di funzionalit\`a di programmazione delle operazioni, sia essa una programmazione singola (``eseguire bonifico a tale beneficiario il tale giorno'') o una programmazione periodica (``eseguire un certo bonifico il tale giorno di ogni mese'', ad esempio per il pagamento di una rata).
L'utente pu\`o visualizzare e impostare tali transazioni programmate, e annullarle prima che vengano effettuate.

%Il nostro sistema vuole dare la possibilità alla banca che lo implementi di adottare una politica di ``conferma della transazione'': in uno scenario di e-commerce, una transazione tra un certo \emph{mittente} e un certo \emph{destinatario} viene confermata se il primo conferma di aver ricevuto in modo corretto il bene acquistato dal secondo, annullata se, viceversa, il primo è in grado di dimostrare la non corretta o avvenuta ricezione del bene acquistato.
%Per permettere questa funzionalità, si suddividerà la procedura di pagamento on-line in più step, ognuna delle quali avrà determinate peculiarità.

\subsection{Titoli di credito}

I titoli di credito sono, generalmente, strumenti finanziari mediante i quali i correntisti investono il proprio danaro: possono essere fondi comuni di investimento o azioni.
Ogni titolo di credito deve avere un \emph{benchmark}, calcolato dalla banca e visibile all'utente, che ne specifichi la \emph{volatilità} sul mercato.
Per motivi di sicurezza ed efficienza del sistema, si decide di non permettere l'acquisto diretto di titoli di credito su mercato nazionale e internazionale, ma di lasciare ad un account utente la possibilità di acquisire pacchetti preconfezionati dalla banca stessa.
Tuttavia ogni correntista può, attraverso il proprio account, avere delle dettagliate informazioni sull'andamento in borsa dei titoli che conserva nel suo portafoglio  e gli è lasciata la possibilità di vendere tali titoli nel caso lo ritenga opportuno.

\subsection{Bidding}

\`E possibile per un correntista o un individuo non registrato \emph{fare bidding}, ossia richiedere alla banca:
\begin{itemize}
	\item L'apertura di un conto corrente soggetto a condizioni specifiche.
	\item Il rilascio di una carta di credito soggetta a condizioni specifiche.
	\item La concessione di un mutuo/prestito a tasso e condizioni specifiche.
\end{itemize}

In base a delle regole definite dai dirigenti la proposta pu\`o essere:
\begin{itemize}
	\item approvata automaticamente dal sistema;
	\item inoltrata a un dipendente della banca per l'approvazione;
	\item rifiutata automaticamente.
\end{itemize}

Un possibile sistema di regole per l'approvazione dei bidding \`e l'individuazione di \emph{aree di risposta} basate sui parametri personalizzabili, come costo mensile di una carta di credito, tetto di spesa massima mensile, capacit\`a di prelievo allo sportello, etc.
Un esempio di come identificare queste aree di risposta \`e dato in figura \ref{fig:bidding}:
\begin{itemize}
	\item la zona compresa fra le linee tratteggiate verdi indica le condizioni (in questo caso una coppia di valori ``spesa massima mensile'' e ``costo mensile'' per una carta di credito) approvate automaticamente;
	\item la zona compresa fra le linee tratteggiate arancioni indica le condizioni soggette a verifica del dirigente della filiale prima dell'accettazione o del rifiuto;
	\item la zona compresa fra le linee tratteggiate rosse indica le condizioni rifiutate automaticamente.
		Le motivazioni di un rifiuto automatico possono essere differenti: un tetto di spesa mensile troppo alto a fronte di un costo mensile troppo basso potrebbe essere economicamente svantaggioso per la banca, mentre un tetto di spesa basso associato a un costo mensile alto potrebbe portare la banca a violare le normative vigenti a tutela dei suoi clienti.
\end{itemize}

\begin{figure}
	\resizebox{\columnwidth}{!}{
	\begin{tikzpicture}
		\begin{axis}[
			ymin=1,
			scale only axis,
			width=\columnwidth,
			ticks=none,
			domain=0:10,
			xlabel={Spesa massima mensile},
			ylabel={Costo mensile},
			axis lines=left,
			samples=200
        ]
		    \path[name path=axis] (axis cs:0,0) -- (axis cs:1,0);
		    \addplot[green] {x};
		    \addplot[green, dashed] {x-1};
		    \addplot[green, dashed] {x+1};
		    \addplot[orange, dashed] {x-2};
		    \addplot[orange, dashed] {x+2};
		    \addplot[red, dashed] {x-3};
		    \addplot[red, dashed] {x+3};
%   			\node [] at (axis cs:0.2,0) {basso};
%   			\node [] at (axis cs:0.6,0.3) {medio};
%   			\node [] at (axis cs:1.0,0.6) {alto};
%   			\node [] at (axis cs:1.4,1.2) {massimo};			
		\end{axis}
	\end{tikzpicture}
	}
	\caption{Un esempio di bidding: confronto fra spesa massima mensile di una carta di credito e costo di mantenimento della stessa.}
	\label{fig:bidding}
\end{figure}

Il sistema di bidding pu\`o portare diversi vantaggi alla banca che lo adotti:
\begin{itemize}
	\item fidelizzazione dell'utenza: un simile meccanismo permette alla banca di venire incontro alle esigenze di clienti di lunga data o con alta giacenza media mensile, che risultano essere ``buoni'' secondo le metriche della dirigenza, fornendo una possibilit\`a di interazione maggiore all'utente;
	\item contrasto della concorrenza: un cliente della banca potrebbe ricevere un'offerta conveniente da un istituto concorrente, ma scoprire autonomamente che un'offerta simile \`e disponibile anche presso la sua banca attuale;
	\item riduzione delle spese: automatizzando parte della ricerca dei contratti si riduce il carico di lavoro che i dipendenti della banca devono subire.
\end{itemize}

Un sistema di bidding presenta dei rischi, fra cui:
\begin{itemize}
	\item il sistema per stabilire i parametri di accettazione automatica potrebbe essere poco intuitivo e non fornire le metriche giuste, e il dirigente che se ne occupa viene portato a scelte che vanno contro gli interessi del suo istituto;
	\item il sistema di bidding potrebbe essere troppo complesso per l'utenza della banca e non venir utilizzato.
\end{itemize}
Tutti questi rischi devono essere gestiti correttamente durante lo sviluppo del sistema.

%TODO pubblicità contestuale on-demand di bidding?

\subsection{Home Trading}

HBS deve implementare una gestione (seppur parziale) dei portafogli fondi dei clienti della banca.

La soluzione individuata, chiamata \emph{Home Trading}, ha le seguenti caratteristiche:
\begin{itemize}
	\item La banca che sfrutta il sistema di \emph{Home Trading} confeziona \emph{pacchetti di investimento} contenenti un certo numero di titoli azionari.
	I pacchetti di investimento vengono proposti ai clienti della banca tramite HBS.
	\item I clienti possono visualizzare informazioni sull'andamento passato dei titoli azionari contenuti in un pacchetto di investimento, assieme a una valutazione del rischio effettuata dal dipendente responsabile del confezionamento del pacchetto.
	I clienti possono acquistare un pacchetto di investimento per un periodo di tempo predeterminato.
	\item A ogni pacchetto di investimento \`e associata una soglia minima (inferiore al prezzo di acquisto) di \emph{valore garantito} del pacchetto.
	Qualora un cliente di HBS voglia vendere un pacchetto di investimento prima del tempo predeterminato, ricever\`a dalla banca il valore garantito del pacchetto, indipendentemente dall'andamento delle azioni contenute nel pacchetto.
	\item Al termine del periodo di possesso del pacchetto i clienti di HBS ricevono dalla banca una somma commisurata alle perdite o ai guadagni delle azioni contenute nel pacchetto.
	Il valore del pacchetto non pu\`o scendere sotto il \emph{valore garantito} stabilito dalla banca, ossia in caso di perdita delle azioni contenute nel pacchetto la perdita del cliente che ha effettuato l'invesitmento \`e sempre minore o uguale alla differenza fra prezzo originale del pacchetto e valore garantito del pacchetto.
\end{itemize}

Il sistema di \emph{Home Trading} proposto permette di gestire i portafogli fondi dei clienti della banca, senza incorrere nell'elevata complessit\`a caratterizzante sistemi di \emph{trading} professionali.
