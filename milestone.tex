\section{Fasi del progetto e milestone}

Il processo software adottato \`e il Rational Unified Process.
Le fasi previste da RUP sono quattro:
\begin{description}
	\item[Inception]
	La prima fase definisce l'ambito del progetto, ne attesta la fattibilit\`a o infattibilit\`a, stabilisce i requisiti fondamentali e individua gli use-case e i rischi principali.
	\item[Elaboration]
	La seconda fase approfondisce quanto stabilito dalla prima fase, cerca inconsistenze ed errori per correggerli, inizia lo sviluppo del sistema.
	Vengono approfonditi e formalizzati i requisiti e gli use-case.
	Viene strutturata nel dettaglio l'architettura del sistema, e se ne inizia l'implementazione.
	\item[Construction]
	\item[Transition]
\end{description}

Ciascuna fase \`e suddivisa in una o pi\`u iterazioni successive.
Ciascuna fase e ciascuna iterazione hanno delle \emph{milestone} che permettono di stabilirne rigorosamente l'avvenuto termine.
Le \emph{milestone} sono quindi insiemi di condizioni che devono essere soddisfatte contemporaneamente.

\subsection{Inception}

\paragraph{Documentazione prodotta}
\begin{itemize}
	\item Analisi del contesto e studio di fattibilit\`a
	\item Documento dei requisiti
	\item Piano di progetto
\end{itemize}

\paragraph{Milestone}
\begin{itemize}
	\item L'intera documentazione prevista per la fase di Inception \`e stata prodotta.
	\item Il direttore del progetto approva la documentazione prodotta.
\end{itemize}

\subsubsection{I iterazione}

La prima iterazione prevede la produzione di una versione iniziale dei documenti del progetto, non necessariamente completi.
Se necessario i documenti saranno soggetti a ulteriori iterazioni.

\paragraph{Milestone}
\begin{itemize}
	\item L'analisi del contesto e lo studio di fattibilit\`a sono stati effettuati.
	\item La definizione dei requisiti \`e stata terminata, i business case principali sono stati definiti.
	\item La prima versione dei documenti \`e stata consegnata al capo progetto.
\end{itemize}

\subsection{Elaboration}

\paragraph{Documentazione prodotta}
\emph{da determinare}

\paragraph{Milestone}
\begin{itemize}
	\item L'intera documentazione prevista per la fase di Elaboration \`e stata prodotta.
	\item I requisiti sono stati individuati e analizzati quasi interamente.
	\item L'architettura del sistema \`e stata stabilita approfonditamente.
	\item L'architettura del sistema soddisfa i requisiti e i casi di business.
	\item Il direttore del progetto approva la documentazione prodotta.
\end{itemize}

\subsection{Construction}

\paragraph{Documentazione prodotta}
\emph{da determinare}

\paragraph{Milestone}
\emph{da determinare}


\subsection{Transition}

\paragraph{Documentazione prodotta}
\emph{da determinare}

\paragraph{Milestone}
\emph{da determinare}
